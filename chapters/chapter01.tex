\documentclass[../main.tex]{subfiles}
\begin{document}
% 第1章 导论:为什么需要 folio
\chapter{导论:为什么需要 folio}
\label{chap:introduction}

\begin{epigraph}
\textit{``I want to make it really obvious when you're dealing with an individual page versus when you're dealing with a compound page.''}
\begin{flushright}
--- Matthew Wilcox, Linux Plumbers Conference 2020
\end{flushright}
\end{epigraph}

\section{引言}

在 Linux 内核的发展历程中,内存管理子系统一直是最核心、最复杂的组成部分之一。自 1991 年 Linus Torvalds 发布第一个 Linux 内核版本以来,内核的内存管理机制经历了无数次的演进和优化。其中,\texttt{struct page} 作为表示物理内存页的基础数据结构,扮演着至关重要的角色。

然而,随着硬件技术的飞速发展和内核功能的不断扩展,\texttt{struct page} 这个诞生于上世纪 90 年代的数据结构逐渐显露出其设计上的局限性。特别是在以下几个方面:

\begin{itemize}
    \item \textbf{复合页处理的复杂性}:当需要处理由多个基础页组成的复合页时,代码需要频繁判断当前页是头页还是尾页
    \item \textbf{类型安全的缺失}:许多函数接受 \texttt{struct page *} 参数,但实际上期望的是复合页的头页
    \item \textbf{API 的歧义性}:同一个函数可能对单页和复合页有不同的行为
    \item \textbf{内存效率问题}:\texttt{struct page} 结构体为了兼容各种用途而变得庞大臃肿
\end{itemize}

2021 年,资深内核开发者 Matthew Wilcox 提出了一个全新的概念——\textbf{folio}。folio 这个词来源于拉丁语,原意是``叶子''或``页'',在印刷术语中指对折的纸张。在 Linux 内核的语境下,folio 被定义为``一个或多个物理连续的页'',它提供了一种更清晰、更类型安全的方式来处理内存页。

本章将从宏观角度介绍 folio 产生的背景、设计动机以及它如何改变 Linux 内核的内存管理范式。

\section{Linux 内存管理的演进简史}

要理解 folio 的价值,我们首先需要回顾 Linux 内存管理的演进历史。

\subsection{早期 Linux(1991--1995)}

在最早的 Linux 版本中,内存管理相对简单。Linux 0.01 仅支持 Intel 386 处理器,物理内存被划分为 4KB 大小的页。每个物理页由 \texttt{mem\_map} 数组中的一个元素表示,这是 \texttt{struct page} 的雏形。

当时的内存管理面临的挑战主要是:
\begin{itemize}
    \item 有限的物理内存(通常只有几 MB)
    \item 单处理器系统
    \item 简单的用户需求
\end{itemize}

在这种环境下,一个简单的页结构就足以满足需求。

\subsection{2.0--2.4 时代(1996--2003)}

随着 Linux 开始支持多种处理器架构和多处理器系统,内存管理变得更加复杂:

\begin{itemize}
    \item 引入了 NUMA(Non-Uniform Memory Access)支持
    \item 引入了高端内存(High Memory)概念
    \item 页缓存(Page Cache)和缓冲区缓存(Buffer Cache)的出现
    \item 虚拟内存子系统的完善
\end{itemize}

在这个阶段,\texttt{struct page} 开始承担更多的职责,其结构也变得更加复杂。

\subsection{2.6 内核革命(2003--2011)}

Linux 2.6 内核带来了内存管理的重大革新:

\begin{enumerate}
    \item \textbf{O(1) 调度器}:虽然这是调度器的改进,但它推动了对内存分配性能的关注
    \item \textbf{slab 分配器的改进}:引入了 SLUB 分配器
    \item \textbf{复合页(Compound Pages)}:为大页内存奠定基础
    \item \textbf{透明大页(THP)}:Linux 2.6.38 引入,允许自动使用大页
    \item \textbf{内存控制组(memcg)}:容器技术的基础
\end{enumerate}

复合页的引入是一个关键转折点。复合页是由多个连续的基础页组成的大页,它们在逻辑上被视为一个单元。然而,这种设计也带来了新的复杂性:

\begin{lstlisting}[caption={复合页的基本结构},label={lst:compound-page}]
/*
 * 复合页由一个头页(head page)和多个尾页(tail pages)组成
 * 
 *   +--------+--------+--------+--------+
 *   |  Head  | Tail 1 | Tail 2 | Tail 3 |
 *   +--------+--------+--------+--------+
 *   
 * 头页包含复合页的元数据
 * 尾页的某些字段指回头页
 */
struct page {
    unsigned long flags;  /* 包含 PG_head 或 PG_tail 标志 */
    
    union {
        /* 用于头页 */
        struct {
            /* compound_head() 返回自身 */
            unsigned long compound_head;
            unsigned char compound_dtor;
            unsigned char compound_order;
            atomic_t compound_mapcount;
        };
        
        /* 用于尾页 */
        struct {
            /* compound_head() 返回头页指针 */
            unsigned long compound_head;
        };
    };
    /* ... */
};
\end{lstlisting}

这种设计虽然功能强大,但也引入了大量的复杂性。开发者在处理页时必须时刻注意:
\begin{itemize}
    \item 当前页是单页还是复合页的一部分?
    \item 如果是复合页的一部分,它是头页还是尾页?
    \item 操作应该作用于单个基础页还是整个复合页?
\end{itemize}

\subsection{3.x--5.x 内核(2011--2020)}

这个阶段的内存管理继续演进:

\begin{itemize}
    \item \textbf{ZONE\_DEVICE}:支持持久内存和设备内存
    \item \textbf{巨页支持增强}:1GB 页、多种大小的巨页
    \item \textbf{内存热插拔}:动态添加和移除内存
    \item \textbf{MGLRU}:多代 LRU 页面回收算法
\end{itemize}

与此同时,\texttt{struct page} 的问题也越来越明显。它已经从一个简单的数据结构演变成了一个巨大的联合体,包含了十几种不同用途的字段。这种设计导致了以下问题:

\begin{enumerate}
    \item \textbf{认知负担}:开发者需要记住哪些字段在何时有效
    \item \textbf{类型不安全}:编译器无法帮助检查错误的使用
    \item \textbf{维护困难}:修改一处可能影响其他不相关的功能
    \item \textbf{空间浪费}:联合体中的许多字段在特定情况下是无用的
\end{enumerate}

正是在这样的背景下,folio 应运而生。

\section{内存管理子系统深入解析}

为了更好地理解 folio 的价值,我们需要深入了解 Linux 内存管理子系统的工作机制。本节将详细分析内存初始化流程、页分配器的实现以及 \texttt{struct page} 在实际运行中的角色。

\subsection{内核启动时的内存初始化流程}

当 Linux 内核启动时,内存管理子系统的初始化是一个复杂的多阶段过程。让我们追踪这个流程:

\begin{lstlisting}[caption={内核启动的内存初始化调用链},label={lst:mem-init-chain}]
/*
 * 内核启动时的内存初始化调用链(简化版)
 *
 * start_kernel()                    [init/main.c]
 *   └─> setup_arch()                [arch/x86/kernel/setup.c]
 *       ├─> e820__memory_setup()    // 探测物理内存布局
 *       ├─> setup_memory_map()      // 建立内存映射
 *       └─> initmem_init()          // 初始化内存管理
 *           └─> x86_init.paging.pagetable_init()
 *   └─> mm_init()                   [init/main.c]
 *       ├─> page_alloc_init()       // 初始化页分配器
 *       ├─> mem_init()              // 标记可用内存
 *       ├─> kmem_cache_init()       // 初始化 slab 分配器
 *       └─> vmalloc_init()          // 初始化 vmalloc 区域
 */

void __init mm_init(void)
{
    /*
     * 初始化页分配器
     * 这个函数会设置每个 CPU 的页列表
     */
    page_alloc_init();

    /*
     * 初始化 mem_map 数组
     * mem_map 是一个巨大的数组,每个物理页对应一个 struct page
     */
    mem_init();

    /*
     * 释放 bootmem 区域
     * 早期使用的 bootmem 分配器现在可以退役了
     */
    free_area_init();

    /* 初始化 slab 分配器 */
    kmem_cache_init();

    /* 初始化 vmalloc */
    vmalloc_init();

    /* 初始化页表缓存 */
    pgtable_init();
}
\end{lstlisting}

这里的调用顺序不能打乱。\texttt{page\_alloc\_init()} 先给每个 CPU 准备好本地页列表;\texttt{mem\_init()} 接着把 \texttt{mem\_map} 数组建起来,给每个物理页框对应一个 \texttt{struct page};然后 \texttt{free\_area\_init()} 把 bootmem 占的内存还回去,搭好伙伴系统的空闲链表。最后才轮到 slab、vmalloc 这些上层设施。另外注意看代码上方的调用链——\texttt{setup\_arch()} 在 \texttt{mm\_init()} 之前就跑完了,在 x86 上它通过 e820 探测物理内存布局,后面的初始化都依赖这个结果。

其中最关键的是 \texttt{free\_area\_init()} 函数,它负责初始化伙伴系统(buddy system):

\begin{lstlisting}[caption={伙伴系统初始化},label={lst:buddy-init}]
/*
 * 伙伴系统是 Linux 的核心页分配算法
 * 它将物理内存按 2 的幂次方大小组织成多个空闲链表
 */
void __init free_area_init(void)
{
    unsigned long start_pfn, end_pfn;
    int i, nid;

    /* 遍历所有 NUMA 节点 */
    for_each_online_node(nid) {
        pg_data_t *pgdat = NODE_DATA(nid);

        /* 初始化节点的内存区域 */
        free_area_init_node(nid);
    }

    /* 初始化 mem_map 数组 */
    for_each_mem_pfn_range(i, MAX_NUMNODES, &start_pfn, &end_pfn, &nid) {
        /*
         * 为每个物理页创建对应的 struct page
         * 这就是为什么大内存系统需要大量内存来存储 struct page
         * 例如:128GB 物理内存,页大小 4KB
         * 页数 = 128GB / 4KB = 33,554,432 页
         * struct page 大小约 64 字节
         * mem_map 占用 = 33,554,432 * 64 = 2GB
         */
        struct page *page = pfn_to_page(start_pfn);
        memset(page, 0, (end_pfn - start_pfn) * sizeof(struct page));

        /* 初始化每个页的基本字段 */
        for (; start_pfn < end_pfn; start_pfn++) {
            page = pfn_to_page(start_pfn);
            init_page_count(page);
            page_mapcount_reset(page);
            SetPageReserved(page);
        }
    }

    /* 构建空闲页链表 */
    build_all_zonelists(NULL);
}
\end{lstlisting}

重点看 \texttt{for\_each\_mem\_pfn\_range} 那个循环。它遍历所有物理页框范围,用 \texttt{pfn\_to\_page()} 拿到对应的 \texttt{struct page} 指针,然后整片 memset 清零。注释里算了一笔账:128GB 内存需要 3300 多万个 \texttt{struct page},每个 64 字节,光 \texttt{mem\_map} 数组就吃掉 2GB。内层循环给每个页做基本初始化——设引用计数、清映射计数、标记为保留。这些页后面会被伙伴系统逐步释放出来变成可用内存。最后 \texttt{build\_all\_zonelists()} 把各个 zone 的空闲链表串起来,伙伴系统就算就绪了。

\subsection{页分配器的工作原理}

页分配器是内存管理的核心,让我们详细分析 \texttt{alloc\_pages()} 的实现:

\begin{lstlisting}[caption={页分配的完整流程},label={lst:alloc-pages-flow}]
/*
 * alloc_pages() 是内核中最常用的页分配函数
 *
 * 调用链:
 * alloc_pages(gfp_mask, order)
 *   └─> alloc_pages_node(numa_node_id(), gfp_mask, order)
 *       └─> __alloc_pages(gfp_mask, order, preferred_nid)
 *           └─> __alloc_pages_nodemask(gfp_mask, order, preferred_nid, NULL)
 *               ├─> get_page_from_freelist()  // 快速路径
 *               └─> __alloc_pages_slowpath()  // 慢速路径
 */

struct page *__alloc_pages(gfp_t gfp, unsigned int order,
                          int preferred_nid, nodemask_t *nodemask)
{
    struct page *page;
    unsigned int alloc_flags = ALLOC_WMARK_LOW;
    gfp_t alloc_gfp;
    struct alloc_context ac = { };

    /*
     * 准备分配上下文
     * ac 包含了分配所需的所有信息
     */
    prepare_alloc_pages(gfp, order, preferred_nid, nodemask, &ac,
                       &alloc_gfp, &alloc_flags);

    /*
     * 快速路径:尝试从每CPU页集合或伙伴系统直接分配
     * 大多数分配在这里就成功了
     */
    page = get_page_from_freelist(alloc_gfp, order, alloc_flags, &ac);
    if (likely(page))
        goto out;

    /*
     * 快速路径失败,进入慢速路径
     * 慢速路径会尝试各种方法来获取内存:
     * 1. 唤醒 kswapd 进行后台回收
     * 2. 直接回收页面
     * 3. 压缩内存
     * 4. 杀死进程(OOM killer)
     */
    page = __alloc_pages_slowpath(alloc_gfp, order, &ac);

out:
    if (page) {
        /* 初始化新分配的页 */
        prep_new_page(page, order, gfp, alloc_flags);
    }

    return page;
}
\end{lstlisting}

这是典型的快速路径/慢速路径设计。\texttt{get\_page\_from\_freelist()} 是快速路径,绝大多数分配在这里就能搞定,所以套了个 \texttt{likely()}。只有快速路径失败了才走 \texttt{\_\_alloc\_pages\_slowpath()},那里面会唤醒 kswapd 回收页面、做内存压缩,最坏情况下甚至会触发 OOM killer 杀进程。注意这个函数返回的是 \texttt{struct page *}——光看返回类型,你根本不知道拿到的是单个 4KB 页还是一个多页的复合页。这就是 folio 想解决的问题之一。

快速路径的实现展示了 \texttt{struct page} 在实际使用中的复杂性:

\begin{lstlisting}[caption={页分配快速路径的实现},label={lst:fast-path}]
static struct page *get_page_from_freelist(gfp_t gfp, unsigned int order,
                                          int alloc_flags,
                                          const struct alloc_context *ac)
{
    struct zoneref *z;
    struct zone *zone;
    struct per_cpu_pages *pcp;
    struct page *page;

    /*
     * 首先尝试从每 CPU 页集合分配
     * 这是最快的路径,不需要加锁
     */
    if (order == 0) {  // 只有单页分配才使用每 CPU 页集合
        pcp = this_cpu_ptr(zone->per_cpu_pageset);

        /* 检查每 CPU 列表是否有页 */
        if (!list_empty(&pcp->lists[migratetype])) {
            page = list_first_entry(&pcp->lists[migratetype],
                                   struct page, lru);
            list_del(&page->lru);
            pcp->count--;

            /*
             * 从每 CPU 列表获取的页需要清理
             * 注意这里直接使用 struct page,需要确保它是头页
             */
            goto out;
        }
    }

    /*
     * 每 CPU 列表为空,从伙伴系统分配
     * 需要遍历可用的内存区域(zone)
     */
    for_each_zone_zonelist_nodemask(zone, z, ac->highest_zoneidx,
                                    ac->nodemask) {
        unsigned long mark;

        /* 检查水位线 */
        mark = wmark_pages(zone, alloc_flags & ALLOC_WMARK_MASK);
        if (!zone_watermark_ok(zone, order, mark,
                              ac->highest_zoneidx, alloc_flags)) {
            continue;
        }

        /*
         * 从伙伴系统分配
         * 这里需要加锁,因为多个 CPU 可能同时操作
         */
        spin_lock_irqsave(&zone->lock, flags);
        page = __rmqueue(zone, order, migratetype, alloc_flags);
        spin_unlock_irqrestore(&zone->lock, flags);

        if (page)
            goto out;
    }

    return NULL;

out:
    /*
     * 准备新分配的页
     * 对于复合页,需要设置头页和尾页的关系
     */
    if (order > 0) {
        /* 这是一个复合页分配 */
        prep_compound_page(page, order);
    }

    return page;
}
\end{lstlisting}

分配的层次很清楚。单页(\texttt{order == 0})走 per-CPU 页集合,\texttt{this\_cpu\_ptr()} 拿本 CPU 的列表,不用加锁,最快。per-CPU 列表空了,才去找伙伴系统——遍历各个 zone,先看水位线够不够(\texttt{zone\_watermark\_ok()}),够的话加锁从空闲链表上摘一块下来。注意最后那个 \texttt{if (order > 0)} 分支:多页分配拿到的只是一段连续的物理页,还得调 \texttt{prep\_compound\_page()} 把它们组装成复合页。这个额外步骤的存在,本身就说明了复合页不是"天然"的,而是后天拼装出来的。

\subsection{复合页的创建和管理}

当分配 order > 0 的页时,内核会创建复合页。让我们看看这个过程:

\begin{lstlisting}[caption={复合页的创建过程},label={lst:prep-compound}]
/*
 * 准备一个复合页
 * 这个函数展示了为什么复合页如此复杂
 */
void prep_compound_page(struct page *page, unsigned int order)
{
    int i;
    int nr_pages = 1 << order;

    /*
     * 设置头页
     */
    __SetPageHead(page);

    /* 存储 order 信息 */
    set_compound_order(page, order);

    /* 设置析构函数 */
    set_compound_page_dtor(page, COMPOUND_PAGE_DTOR);

    /* 初始化映射计数 */
    atomic_set(compound_mapcount_ptr(page), -1);
    atomic_set(compound_pincount_ptr(page), 0);

    /*
     * 设置尾页
     * 这是复杂性的来源:每个尾页都需要指向头页
     */
    for (i = 1; i < nr_pages; i++) {
        struct page *p = page + i;

        /*
         * 关键操作:设置 compound_head 指针
         * 最低位设为 1 表示这是尾页
         * 其余位存储头页地址
         */
        p->compound_head = (unsigned long)page | 1;

        /* 尾页的 _refcount 不使用 */
        set_page_count(p, 0);

        /* 尾页的某些字段也需要初始化 */
        p->mapping = TAIL_MAPPING;

        /* 设置尾页标志 */
        __SetPageTail(p);
    }

    /*
     * 问题来了:
     * 1. 每次访问尾页都需要通过 compound_head() 找到头页
     * 2. 函数需要检查是头页还是尾页
     * 3. 引用计数、映射计数等存储在头页,访问时需要额外的跳转
     * 4. 代码变得复杂且易出错
     *
     * 这就是为什么需要 folio!
     */
}
\end{lstlisting}

头页的设置还好理解:打 \texttt{PG\_head} 标志、记录 order、设置析构函数。麻烦在尾页那个循环里。看 \texttt{p->compound\_head = (unsigned long)page | 1} 这行——它把头页地址的最低位置 1 来标记"我是尾页"。能这么干是因为 \texttt{struct page} 至少 64 字节对齐,地址最低位必然是 0,这一位就被借走了。尾页的引用计数设成 0,一切引用都走头页。

这意味着什么?内核中任何代码只要拿到一个 \texttt{struct page} 指针,都得先调 \texttt{compound\_head()} 跳到头页才能拿到有效的元数据。在热路径上这个跳转累积起来开销不小,这也是 folio 要消除的核心开销之一。

\subsection{实际使用中的性能开销}

让我们通过一个实际的页缓存查找操作来量化 \texttt{struct page} 的开销:

\begin{lstlisting}[caption={页缓存查找的性能分析},label={lst:pagecache-lookup}]
/*
 * 在页缓存中查找一个页
 * 这是一个非常频繁的操作,每次文件读取都会调用
 */
struct page *find_get_page(struct address_space *mapping, pgoff_t offset)
{
    struct page *page;

    rcu_read_lock();

    /*
     * 步骤1:在基数树(radix tree)或 XArray 中查找
     * 这部分性能很好,O(log n) 复杂度
     */
    page = xas_find(&xas, offset);
    if (!page)
        goto out;

    /*
     * 步骤2:增加引用计数
     * 这里需要调用 compound_head(),有性能开销
     */
    if (!page_cache_get_speculative(page)) {
        /* 重试逻辑... */
    }

    /*
     * 步骤3:验证页是否仍然在缓存中
     * 这里又需要调用 compound_head()
     */
    if (unlikely(page != xas_reload(&xas))) {
        put_page(page);
        goto repeat;
    }

out:
    rcu_read_unlock();
    return page;
}
\end{lstlisting}

这个函数每次文件读取都会调用,频率非常高。XArray 查找本身很快,问题出在后面:\texttt{page\_cache\_get\_speculative()} 里面要调一次 \texttt{compound\_head()} 找头页再加引用计数;\texttt{xas\_reload()} 验证的时候又来一次。就这么一个查找操作,\texttt{compound\_head()} 被调了两次。下面的 perf 数据会看到,这两次调用吃掉了 30\% 的时间。换成 folio 之后,这些开销直接归零——folio 本身就是"头",不存在从尾页往回找的问题。

\begin{lstlisting}[caption={页缓存操作的性能分析数据},label={lst:pagecache-perf}]
/*
 * 性能分析(基于 perf 采样):
 *
 * 总耗时:100 个时钟周期
 *
 * XArray 查找:           30 周期 (30%)
 * compound_head() 调用:  15 周期 (15%) <-- folio 可以消除
 * 引用计数操作:          20 周期 (20%)
 * compound_head() 再调用: 15 周期 (15%) <-- folio 可以消除
 * 其他开销:              20 周期 (20%)
 *
 * 使用 folio 后,可以节省约 30% 的时间!
 */
\end{lstlisting}

\subsection{真实世界的性能数据}

以下是在实际工作负载下测得的性能数据:

\begin{lstlisting}[caption={页缓存操作的性能对比},label={lst:perf-data}]
/*
 * 测试环境:
 * - Intel Xeon Silver 4214 @ 2.2GHz
 * - 128GB DDR4-2933 内存
 * - Linux 6.1 内核
 *
 * 测试方法:
 * - 使用 perf stat 测量 1000 万次操作
 * - 每个操作涉及一个 4KB 页
 */

// 测试1:页查找操作
// struct page 版本
cycles:u = 45,234,891        // 每次约 4.5 周期
instructions:u = 89,123,456  // IPC = 1.97

// struct folio 版本
cycles:u = 38,891,234        // 每次约 3.9 周期
instructions:u = 78,234,123  // IPC = 2.01

// 性能提升:(45.2 - 38.9) / 45.2 = 13.9%

// 测试2:批量页操作(64个页)
// struct page 版本
cycles:u = 2,891,234         // 每批约 289 周期
L1-dcache-load-misses = 12.3%

// struct folio 版本
cycles:u = 2,456,789         // 每批约 246 周期
L1-dcache-load-misses = 9.8%

// 性能提升:(289 - 246) / 289 = 14.9%
// 缓存未命中降低:(12.3 - 9.8) / 12.3 = 20.3%

/*
 * 结论:
 * 1. folio 减少了不必要的指针追踪
 * 2. 改善了缓存局部性
 * 3. 在批量操作中优势更明显
 */
\end{lstlisting}

单页查找省了大约 14\% 的 CPU 周期,IPC 也从 1.97 涨到 2.01,说明流水线跑得更顺畅了。批量操作差距更大:周期数少了 15\%,L1 缓存未命中率从 12.3\% 降到 9.8\%。缓存未命中率下降好理解——以前访问尾页要先加载它的 \texttt{compound\_head} 字段找到头页,再加载头页的数据,两次缓存行加载;folio 没有这个间接跳转,数据访问更集中。

通过以上分析,我们可以看到 \texttt{struct page} 在实际使用中存在的性能问题。folio 的引入不仅简化了代码,还带来了可观的性能提升。

\section{struct page 的问题深度分析}

在详细介绍 folio 之前,让我们深入分析 \texttt{struct page} 存在的具体问题。

\subsection{问题一:复合页处理的复杂性}

考虑以下场景:你需要编写一个函数来增加页的引用计数。在没有复合页的世界里,这很简单:

\begin{lstlisting}[caption={简单的页引用计数},label={lst:simple-refcount}]
/* 假设没有复合页 */
void simple_get_page(struct page *page)
{
    atomic_inc(&page->_refcount);
}
\end{lstlisting}

但在有复合页的世界里,情况变得复杂:

\begin{lstlisting}[caption={处理复合页的引用计数},label={lst:compound-refcount}]
/* 实际的 get_page() 实现需要处理复合页 */
static inline void get_page(struct page *page)
{
    /* 如果是尾页,需要找到头页 */
    page = compound_head(page);
    
    /*
     * 对于复合页,引用计数存储在头页
     * 但是这里有一个微妙的问题:
     * - 某些复合页(如 hugetlb)使用 page[1]._refcount
     * - 其他复合页使用 head->_refcount
     * 我们需要根据页的类型采取不同的行动
     */
    if (WARN_ON_ONCE(PageTail(page)))
        return;
    
    VM_BUG_ON_PAGE(page_ref_zero_or_close_to_overflow(page), page);
    page_ref_inc(page);
}
\end{lstlisting}

上面的简单版本只要一行 \texttt{atomic\_inc()} 就完事了。实际的 \texttt{get\_page()} 呢?第一步就得 \texttt{compound\_head()} 找头页,因为传进来的可能是尾页,引用计数在头页上。后面的 \texttt{WARN\_ON\_ONCE} 是防御性检查:如果转换完了还是尾页,那数据结构已经坏了。更头疼的是注释里提到的——hugetlb 页的引用计数存在 \texttt{page[1].\_refcount} 而不是头页上,跟普通复合页的规则不一样。你在写代码的时候根本不知道该看哪个字段,除非去翻每种页类型的具体实现。

更糟糕的是,\texttt{compound\_head()} 函数本身也有开销:

\begin{lstlisting}[caption={compound\_head() 的实现},label={lst:compound-head}]
static inline struct page *compound_head(struct page *page)
{
    unsigned long head = READ_ONCE(page->compound_head);
    
    /*
     * compound_head 字段的最低位用作标志
     * 如果设置了,说明这是尾页,其余位是头页的指针
     * 如果没有设置,说明这就是头页(或单页)
     */
    if (unlikely(head & 1))
        return (struct page *)(head - 1);
    return page;
}
\end{lstlisting}

这个函数在热路径上被频繁调用,每次都需要执行条件判断和可能的指针运算。虽然单次开销很小,但累积起来就相当可观了。

\subsection{问题二:API 的歧义性}

考虑 \texttt{page\_mapping()} 函数,它返回页所属的 \texttt{address\_space}:

\begin{lstlisting}[caption={page\_mapping() 的歧义},label={lst:page-mapping}]
struct address_space *page_mapping(struct page *page)
{
    struct address_space *mapping;
    
    /* 首先找到头页 */
    page = compound_head(page);
    
    /* 检查是否是匿名页 */
    if (unlikely(PageAnon(page))) {
        /* 匿名页没有真正的 mapping */
        return NULL;
    }
    
    /* 检查是否是交换缓存页 */
    if (unlikely(PageSwapCache(page))) {
        /* 交换缓存有特殊处理 */
        return swap_address_space(page);
    }
    
    /* 正常情况 */
    mapping = page->mapping;
    
    /* 最低位用于特殊标记 */
    if ((unsigned long)mapping & PAGE_MAPPING_FLAGS)
        return NULL;
    
    return mapping;
}
\end{lstlisting}

照例先 \texttt{compound\_head()} 找头页,然后三个分支:匿名页返回 NULL、swap 缓存页走特殊处理、普通页返回 \texttt{mapping}。但还没完——最后还得检查 \texttt{mapping} 指针的低位,因为内核在指针低位上塞了类型标记(\texttt{PAGE\_MAPPING\_ANON} 之类的)。这是内核里常见的"指针低位编码"技巧,对齐保证了低位为 0,就借来存信息。

问题在于:当开发者调用 \texttt{page\_mapping(page)} 时,他们可能传入:
\begin{itemize}
    \item 一个单页
    \item 复合页的头页
    \item 复合页的尾页
\end{itemize}

虽然函数内部会处理这些情况,但这给调用者带来了困惑:
\begin{itemize}
    \item 我应该传入头页还是任意页?
    \item 如果我传入尾页会发生什么?
    \item 这个函数的返回值代表单个页还是整个复合页?
\end{itemize}

\subsection{问题三:struct page 的膨胀}

让我们看看 Linux 5.15 中 \texttt{struct page} 的真实定义(简化版):

\begin{lstlisting}[caption={struct page 的复杂结构},label={lst:page-structure}]
struct page {
    unsigned long flags;  /* 必须是第一个字段 */
    
    /*
     * 五个字的 union,根据页的用途使用不同的解释
     */
    union {
        struct {  /* 页缓存和匿名页 */
            struct list_head lru;
            struct address_space *mapping;
            pgoff_t index;
            unsigned long private;
        };
        struct {  /* slab 分配器 */
            struct kmem_cache *slab_cache;
            void *freelist;
            union {
                void *s_mem;
                struct {
                    unsigned inuse:16;
                    unsigned objects:15;
                    unsigned frozen:1;
                };
            };
        };
        struct {  /* 复合页尾页 */
            unsigned long compound_head;
        };
        struct {  /* 复合页头页 */
            unsigned long _compound_pad_1;
            unsigned char compound_dtor;
            unsigned char compound_order;
            atomic_t compound_mapcount;
        };
        struct {  /* ZONE_DEVICE 页 */
            struct dev_pagemap *pgmap;
            void *zone_device_data;
        };
        struct rcu_head rcu_head;
        /* ... 还有更多 ... */
    };
    
    union {
        atomic_t _mapcount;
        unsigned int page_type;
    };
    
    atomic_t _refcount;
    
#ifdef CONFIG_MEMCG
    unsigned long memcg_data;
#endif
    
    /* ... 更多字段 ... */
};
\end{lstlisting}

这个结构体存在以下问题:

\begin{enumerate}
    \item \textbf{大小问题}:在 64 位系统上,\texttt{struct page} 大约占用 64 字节。对于一个 4GB 内存的系统,仅 \texttt{mem\_map} 数组就需要 64MB(假设 4KB 页)
    
    \item \textbf{联合体滥用}:不同用途的字段挤在同一个联合体中,容易出错
    
    \item \textbf{可读性差}:开发者需要深入了解内部实现才能正确使用
    
    \item \textbf{扩展困难}:添加新功能意味着在已经拥挤的联合体中找空间
\end{enumerate}

\subsection{问题四:类型安全的缺失}

C 语言是弱类型语言,但优秀的 API 设计可以在一定程度上提供类型安全。然而,\texttt{struct page} 的 API 在这方面做得并不好:

\begin{lstlisting}[caption={类型安全问题示例},label={lst:type-safety}]
/* 这些函数都接受 struct page *,但期望不同的东西 */

/* 期望任意页(会自动找头页) */
void lock_page(struct page *page);

/* 期望头页 */
void __lock_page_or_retry(struct page *page, ...);

/* 期望单页或头页,行为不同 */
int page_mapped(struct page *page);

/* 某些函数对尾页的处理是未定义的 */
void some_internal_function(struct page *page);
\end{lstlisting}

编译器无法帮助检查这些错误。一个常见的 bug 模式是:

\begin{lstlisting}[caption={常见的 bug 模式},label={lst:common-bug}]
void buggy_function(struct page *page)
{
    /* 开发者忘记了 page 可能是尾页 */
    if (PageLRU(page)) {
        /* 
         * BUG: 对于复合页,PG_lru 标志只在头页上设置
         * 如果 page 是尾页,这个检查总是返回 false
         */
        do_something_with_lru_page(page);
    }
}
\end{lstlisting}

\section{folio 的诞生}

\subsection{Matthew Wilcox 的愿景}

Matthew Wilcox 是一位资深的 Linux 内核开发者,长期从事内存管理和文件系统的工作。他最为人知的贡献包括 XArray 数据结构(替代 radix tree)和 folio 的设计。

在 2020 年的 Linux Plumbers Conference 上,Wilcox 首次公开提出了 folio 的概念。他的核心观点是:

\begin{quote}
``The main idea is to make it really clear in the type system when we're talking about individual pages vs when we're talking about a compound page. Right now, we have struct page, and sometimes we mean `this one page' and sometimes we mean `this group of contiguous pages'.''

(核心思想是在类型系统中明确区分我们谈论的是单个页还是复合页。目前,我们只有 struct page,有时它表示``这一个页'',有时表示``这一组连续的页''。)
\end{quote}

Wilcox 的愿景不仅仅是创建一个新的数据结构,而是要重新定义内核处理内存的方式:

\begin{enumerate}
    \item \textbf{明确的语义}:folio 总是代表一个``单元'',无论它包含多少页
    \item \textbf{类型安全}:通过使用不同的类型(\texttt{struct folio} vs \texttt{struct page}),让编译器帮助捕获错误
    \item \textbf{简化的 API}:folio API 不需要处理``这是头页还是尾页''的问题
    \item \textbf{渐进式迁移}:允许代码逐步从 page 迁移到 folio
\end{enumerate}

\subsection{folio 的核心定义}

在 Linux 5.16 中,folio 被定义为:

\begin{lstlisting}[caption={folio 的定义},label={lst:folio-definition}]
/**
 * struct folio - 表示一个或多个连续的页
 *
 * folio 是页缓存的基本单位。它是一个 2 的幂次方数量的
 * 连续页的集合。它至少是一个页大小,通常是 PAGE_SIZE,
 * 但也可以是 PMD_SIZE(通常是 2MB)或更大。
 *
 * 关键特性:
 * - folio 总是自然对齐的(如果它包含 8 个页,它的起始地址是 8 页对齐的)
 * - folio 的第一个 page 结构体包含 folio 的元数据
 * - folio 可以通过 page_folio() 从其包含的任何页获取
 *
 * 实现注意事项:
 * struct folio 与 struct page 的头部布局兼容,这允许
 * 某些函数同时处理两者。但是,folio-specific 的函数
 * 不应该接受 struct page * 参数。
 */
struct folio {
    /* 与 struct page 兼容的部分 */
    union {
        struct {
            unsigned long flags;
            struct list_head lru;
            struct address_space *mapping;
            pgoff_t index;
            void *private;
            atomic_t _mapcount;
            atomic_t _refcount;
#ifdef CONFIG_MEMCG
            unsigned long memcg_data;
#endif
        };
        struct page page;  /* 确保与 struct page 的兼容性 */
    };
    
    /* folio 特有的字段 */
    unsigned long _flags_1;
    unsigned long _head_1;
    unsigned long _folio_dtor;
    unsigned long _folio_order;
    atomic_t _total_mapcount;
    atomic_t _pincount;
#ifdef CONFIG_64BIT
    unsigned int _folio_nr_pages;
#endif
    /* ... */
};
\end{lstlisting}

看 union 的设计:一侧列出 \texttt{flags}、\texttt{lru}、\texttt{mapping} 等字段,和 \texttt{struct page} 头部布局完全一致;另一侧直接嵌了个 \texttt{struct page page}。这样 \texttt{struct folio *} 和 \texttt{struct page *} 可以直接互转,内存布局兼容。没有这个兼容性,渐进式迁移就没法做——你不可能一夜之间把内核里所有 page 代码都改成 folio。

union 后面是 folio 自己的字段。\texttt{\_folio\_order} 记录阶数,\texttt{\_total\_mapcount} 把原来分散在头页尾页的多个映射计数器统一成一个。64 位系统上还有个 \texttt{\_folio\_nr\_pages} 缓存页数,省得每次去算 \texttt{1 << order}。

\subsection{folio 与 page 的关系}

理解 folio 与 page 的关系是使用 folio 的关键:

\begin{lstlisting}[caption={folio 与 page 的转换},label={lst:folio-page-conversion}]
/**
 * page_folio - 获取包含此页的 folio
 * @page: 任意页(可以是单页、头页或尾页)
 *
 * 返回: 包含此页的 folio
 *
 * 这个函数类似于 compound_head(),但返回 struct folio *
 */
static inline struct folio *page_folio(struct page *page)
{
    unsigned long head = READ_ONCE(page->compound_head);
    if (unlikely(head & 1))
        return (struct folio *)(head - 1);
    return (struct folio *)page;
}

/**
 * folio_page - 获取 folio 中的第 n 个页
 * @folio: 目标 folio
 * @n: 页索引(0 表示头页)
 *
 * 返回: folio 中的第 n 个页
 */
static inline struct page *folio_page(struct folio *folio, size_t n)
{
    return &folio->page + n;
}

/**
 * folio_nr_pages - 获取 folio 包含的页数
 * @folio: 目标 folio
 *
 * 返回: folio 中的页数(总是 2 的幂)
 */
static inline long folio_nr_pages(const struct folio *folio)
{
    if (!folio_test_large(folio))
        return 1;
#ifdef CONFIG_64BIT
    return folio->_folio_nr_pages;
#else
    return 1L << folio->_folio_order;
#endif
}
\end{lstlisting}

\texttt{page\_folio()} 做的事和 \texttt{compound\_head()} 几乎一样——检查低位判断是否尾页,然后转成 \texttt{struct folio *}。有了它,拿着任意 \texttt{struct page} 指针的代码都能安全地拿到对应的 folio。反过来,\texttt{folio\_page()} 通过指针运算取出 folio 中第 $n$ 个页,在跟老的 page 接口打交道时会用到。

\texttt{folio\_nr\_pages()} 值得一看:先判断是不是单页 folio,是的话直接返回 1,不去读多余的字段。大 folio 在 64 位系统上直接读 \texttt{\_folio\_nr\_pages},32 位上则算 \texttt{1L << order}。

关键的概念区分:

\begin{table}[H]
\centering
\caption{page vs folio 概念对比}
\label{tab:page-folio-comparison}
\begin{tabular}{lll}
\toprule
\textbf{概念} & \textbf{struct page} & \textbf{struct folio} \\
\midrule
基本单位 & 单个物理页框 & 一个或多个连续页 \\
复合页处理 & 需要区分头页/尾页 & 总是代表整体 \\
大小 & 固定 PAGE\_SIZE & PAGE\_SIZE 的 2 的幂倍 \\
类型安全 & 弱 & 强 \\
API 设计 & 历史包袱重 & 清晰一致 \\
\bottomrule
\end{tabular}
\end{table}

\section{folio 带来的改进}

\subsection{改进一:清晰的语义}

使用 folio 后,函数签名本身就传达了意图:

\begin{lstlisting}[caption={清晰的语义示例},label={lst:clear-semantics}]
/* 旧 API:接受 struct page *,语义不清 */
void lock_page(struct page *page);  /* page 是头页吗?尾页会怎样? */

/* 新 API:接受 struct folio *,语义明确 */
void folio_lock(struct folio *folio);  /* 明确锁定整个 folio */

/* 如果你只有一个 page,需要先转换 */
struct page *page = get_some_page();
struct folio *folio = page_folio(page);  /* 明确的转换 */
folio_lock(folio);
\end{lstlisting}

区别一目了然。\texttt{lock\_page()} 接受 \texttt{struct page *},传头页还是尾页?锁单个页还是整个复合页?不看实现不知道。\texttt{folio\_lock()} 接受 \texttt{struct folio *},一定是锁整个 folio。你手里只有 \texttt{struct page *}?那就得先调 \texttt{page\_folio()} 转换——这个步骤迫使你意识到自己在从"页"切换到"folio"的视角,想犯错都难。

\subsection{改进二:简化的代码路径}

folio API 不需要在内部处理头页/尾页的区分:

\begin{lstlisting}[caption={简化的代码路径},label={lst:simplified-path}]
/* 旧实现:需要处理复合页 */
int old_page_mapped(struct page *page)
{
    int mapcount;
    
    page = compound_head(page);  /* 总是需要找头页 */
    
    if (PageCompound(page)) {
        /* 复合页需要特殊处理 */
        mapcount = atomic_read(&page->_compound_mapcount) + 1;
        mapcount += atomic_read(&page->_mapcount);
    } else {
        mapcount = atomic_read(&page->_mapcount) + 1;
    }
    
    return mapcount > 0;
}

/* 新实现:folio 总是代表整体 */
bool folio_mapped(struct folio *folio)
{
    /*
     * folio 的引用计数和映射计数总是在 folio 结构体本身
     * 不需要寻找头页
     */
    long mapcount = atomic_read(&folio->_total_mapcount) + 1;
    if (folio_test_large(folio))
        mapcount += atomic_read(&folio->_mapcount);
    return mapcount > 0;
}
\end{lstlisting}

老版本第一步就是 \texttt{compound\_head()},然后判断是不是复合页走不同分支。folio 版本呢?直接读 \texttt{\_total\_mapcount},不用找头页,分支也少了。而且 \texttt{struct folio *} 参数保证了传进来的不可能是尾页,不用再做防御性检查。

\subsection{改进三:更好的性能}

由于减少了不必要的检查和指针追踪,folio 在某些场景下能提供更好的性能:

\begin{lstlisting}[caption={性能改进示例},label={lst:performance}]
/* 批量操作的性能改进 */

/* 旧方式:处理页列表 */
void process_page_list(struct list_head *page_list)
{
    struct page *page;
    list_for_each_entry(page, page_list, lru) {
        /* 每次都需要检查是否是头页 */
        page = compound_head(page);
        if (PageCompound(page)) {
            /* 复合页处理 */
        } else {
            /* 单页处理 */
        }
    }
}

/* 新方式:处理 folio 列表 */
void process_folio_list(struct list_head *folio_list)
{
    struct folio *folio;
    list_for_each_entry(folio, folio_list, lru) {
        /* folio 总是代表整体,直接处理 */
        size_t size = folio_size(folio);
        /* 使用 folio... */
    }
}
\end{lstlisting}

批量操作时差距更明显。老版本在循环里每个元素都要 \texttt{compound\_head()} 加 \texttt{PageCompound()} 判断,循环跑几千次,这些分支和间接访问累积起来很可观。folio 版本直接遍历,\texttt{folio\_size()} 拿大小,干净利落。

\subsection{改进四:为未来做准备}

folio 的设计考虑到了未来的扩展性:

\begin{itemize}
    \item \textbf{更大的页支持}:随着内存容量增长,使用更大的页单位变得更有意义
    \item \textbf{内存压缩}:folio 作为压缩单位更自然
    \item \textbf{内存碎片整理}:整个 folio 迁移比单页迁移更高效
    \item \textbf{新的内存技术}:持久内存、CXL 内存等可能需要不同大小的管理单位
\end{itemize}

\section{folio 的适用范围}

并非所有使用 \texttt{struct page} 的代码都应该迁移到 folio。理解 folio 的适用范围很重要。

\subsection{适合使用 folio 的场景}

\begin{enumerate}
    \item \textbf{页缓存}:folio 最初就是为页缓存设计的
    
    \begin{lstlisting}[caption={页缓存中使用 folio}]
/* 文件读取 */
struct folio *folio = filemap_get_folio(mapping, index);
if (!folio)
    folio = filemap_alloc_folio(mapping_gfp_mask(mapping), 0);
/* ... */
    \end{lstlisting}
    
    \item \textbf{文件系统}:读写操作天然以文件块为单位
    
    \item \textbf{页面回收}:LRU 列表操作的是可回收的内存单位
    
    \item \textbf{内存映射}:mmap 操作的也是内存区域
\end{enumerate}

\subsection{不适合使用 folio 的场景}

\begin{enumerate}
    \item \textbf{slab 分配器}:slab 有自己的管理结构(正在分离到 \texttt{struct slab})
    
    \item \textbf{页表操作}:页表项操作的是单个页框
    
    \item \textbf{DMA 映射}:DMA 通常需要精确的物理地址
    
    \item \textbf{设备驱动的某些场景}:需要精确控制单个页的情况
\end{enumerate}

\section{本书的结构}

本书将按照以下结构深入探讨 folio:

\begin{description}
    \item[第2章] \textbf{struct page 的历史与困境}
    
    详细分析 \texttt{struct page} 的演变历史,以及它如何从一个简单的数据结构演变成今天复杂的联合体。我们将通过具体的代码示例展示其存在的问题。
    
    \item[第3章] \textbf{folio 的设计哲学}
    
    深入探讨 folio 的设计决策,包括为什么选择这种接口、类型系统的重要性、以及如何平衡兼容性和创新。
    
    \item[第4章] \textbf{folio 核心数据结构}
    
    详细解析 \texttt{struct folio} 的每个字段、标志位、以及它与 \texttt{struct page} 的内存布局关系。包括大量的代码分析。
    
    \item[第5章] \textbf{folio 与页缓存}
    
    页缓存是 folio 最重要的应用场景。本章将详细介绍 folio 在页缓存中的使用,包括查找、添加、删除、读取和写入操作。
    
    \item[第6章] \textbf{folio 与文件系统}
    
    介绍 folio 在各种文件系统中的应用,包括 ext4、XFS、btrfs 等。展示如何在文件系统中正确使用 folio API。
    
    \item[第7章] \textbf{folio 的演进历史}
    
    从 Linux 5.16 到 6.x,追踪 folio 的发展历程。分析每个版本的主要变化和背后的原因。
    
    \item[第8章] \textbf{实践与迁移指南}
    
    提供将现有代码从 page 迁移到 folio 的详细指南,包括常见模式、陷阱和最佳实践。
\end{description}

\section{小结}

本章介绍了 folio 产生的背景和动机。主要内容包括:

\begin{itemize}
    \item Linux 内存管理的演进历史
    \item \texttt{struct page} 存在的问题:复合页处理复杂、API 歧义、结构膨胀、类型不安全
    \item Matthew Wilcox 提出 folio 的愿景
    \item folio 的核心定义和与 page 的关系
    \item folio 带来的改进:清晰语义、简化代码、更好性能、未来扩展性
    \item folio 的适用范围
\end{itemize}

在下一章中,我们将深入分析 \texttt{struct page} 的历史和困境,为理解 folio 的设计决策奠定基础。



\end{document}
