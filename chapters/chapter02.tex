\documentclass[../main.tex]{subfiles}
\begin{document}
% 第2章 struct page 的历史与困境
\chapter{struct page 的历史与困境}
\label{chap:page-history}

\begin{epigraph}
\textit{``struct page is the most abused data structure in the kernel.''}
\begin{flushright}
--- 匿名内核开发者
\end{flushright}
\end{epigraph}

\section{引言}

要真正理解 folio 的价值,我们必须深入了解 \texttt{struct page} 的历史演变和它所面临的困境。\texttt{struct page} 是 Linux 内核中最古老、最核心的数据结构之一,它的设计深刻影响着整个内存管理子系统。

本章将从源码考古的角度,追溯 \texttt{struct page} 从 Linux 早期版本到现代内核的演变过程,分析其设计决策的历史背景,以及这些决策如何导致了今天的问题。

\section{早期 Linux 的内存管理}

\subsection{Linux 0.01 的内存模型}

1991 年,当 Linus Torvalds 发布 Linux 0.01 时,内存管理还非常原始。让我们看看当时的实现:

\begin{lstlisting}[caption={Linux 0.01 的内存管理(mm/memory.c)}]
/*
 * Linux 0.01 的内存管理
 * 来源: linux-0.01/mm/memory.c
 */

#define LOW_MEM 0x100000           /* 1MB 以下是 BIOS 和内核 */
#define PAGING_MEMORY (15*1024*1024)  /* 可分页内存大小 */
#define PAGING_PAGES (PAGING_MEMORY>>12)  /* 可用页数 */
#define MAP_NR(addr) (((addr)-LOW_MEM)>>12)  /* 地址到页号 */
#define USED 100

static unsigned char mem_map[PAGING_PAGES] = {0,};

/*
 * 获取一个空闲页
 * 从高地址向低地址搜索
 */
unsigned long get_free_page(void)
{
    register unsigned long __res asm("ax");
    __asm__("std ; repne ; scasb\n\t"
            "jne 1f\n\t"
            "movb $1,1(%%edi)\n\t"
            "sall $12,%%ecx\n\t"
            "addl %2,%%ecx\n\t"
            "movl %%ecx,%%edx\n\t"
            "movl $1024,%%ecx\n\t"
            "leal 4092(%%edx),%%edi\n\t"
            "rep ; stosl\n\t"
            "movl %%edx,%%eax\n"
            "1:"
            :"=a" (__res)
            :"0" (0),"i" (LOW_MEM),"c" (PAGING_PAGES),
             "D" (mem_map+PAGING_PAGES-1)
            );
    return __res;
}

/*
 * 释放一个页
 */
void free_page(unsigned long addr)
{
    if (addr < LOW_MEM) return;
    if (addr >= HIGH_MEMORY)
        panic("trying to free nonexistent page");
    addr -= LOW_MEM;
    addr >>= 12;
    if (mem_map[addr]--) return;
    mem_map[addr]=0;
    panic("trying to free free page");
}
\end{lstlisting}

这就是 Linux 内存管理的起点。没有 \texttt{struct page},没有伙伴系统,连个像样的数据结构都没有。\texttt{mem\_map} 就是一个 \texttt{unsigned char} 数组,一个字节对应一个物理页框,字节值就是引用计数。整个可分页内存被硬编码为 15MB,1MB 以下留给 BIOS 和内核。

\texttt{get\_free\_page()} 可能是你见过的最"硬核"的页分配器了:整个分配逻辑写成了一段内联汇编。\texttt{std; repne; scasb} 从数组末尾向前扫描,找到第一个值为 0 的条目。找到后左移 12 位还原成物理地址,再用 \texttt{rep; stosl} 把 4KB 清零。完全不可移植,但在 i386 上跑得飞快。

\texttt{free\_page()} 简单得多,但有个小技巧值得留意:\texttt{if (mem\_map[addr]--) return} 利用了后缀递减的语义。如果计数非零,递减后直接返回;如果已经是 0,递减会下溢,接下来的代码把它重置为 0 并 panic。这是一种很紧凑的双重释放检测。

关键观察:
\begin{itemize}
    \item 没有 \texttt{struct page},只有一个简单的字节数组 \texttt{mem\_map}
    \item 每个页的元数据只有一个字节(引用计数)
    \item 完全针对 i386 架构,使用汇编优化
    \item 没有考虑多处理器、高端内存等复杂情况
\end{itemize}

\subsection{struct page 的诞生}

随着内核的发展,简单的字节数组已经无法满足需求。\texttt{struct page} 在 Linux 2.0 左右开始成形。让我们看看早期版本:

\begin{lstlisting}[caption={Linux 2.0 的 struct page(简化)}]
/*
 * Linux 2.0 时期的 struct page
 * 这是一个简化的重构版本
 */
struct page {
    /* 页的状态标志 */
    unsigned long flags;
    
    /* 引用计数 */
    atomic_t count;
    
    /* 用于各种链表 */
    struct page *next;
    struct page *prev;
    
    /* 关联的 inode(用于页缓存) */
    struct inode *inode;
    
    /* 文件中的偏移 */
    unsigned long offset;
    
    /* 关联的缓冲区头 */
    struct buffer_head *buffers;
};
\end{lstlisting}

这个结构相对简单,但已经开始承担多重职责:
\begin{itemize}
    \item 作为物理页框的描述符
    \item 作为页缓存的一部分
    \item 作为缓冲区缓存的容器
\end{itemize}

\section{struct page 的演变}

\subsection{2.4 内核:复杂性的开始}

Linux 2.4 内核对内存管理做了重大改进,\texttt{struct page} 也随之变得更加复杂:

\begin{lstlisting}[caption={Linux 2.4 的 struct page}]
/*
 * Linux 2.4.x 的 struct page
 * include/linux/mm.h
 */
typedef struct page {
    /* 链表指针,根据页的状态加入不同链表 */
    struct list_head list;
    
    /* 用于页缓存的 mapping */
    struct address_space *mapping;
    
    /* 文件内索引或交换空间标识 */
    unsigned long index;
    
    /* 下一个哈希链表项 */
    struct page *next_hash;
    
    /* 引用计数 */
    atomic_t count;
    
    /* 页标志 */
    unsigned long flags;
    
    /* 用于 LRU 链表 */
    struct list_head lru;
    
    /* 关联的缓冲区 */
    struct buffer_head *buffers;
    
    /* 虚拟地址(仅高端内存) */
    void *virtual;
    
    /* 所属的内存区域 */
    struct zone_struct *zone;
} mem_map_t;
\end{lstlisting}

2.4 内核引入了几个重要概念:
\begin{enumerate}
    \item \textbf{address\_space}:抽象了文件与内存的映射关系
    \item \textbf{内存区域(zones)}:ZONE\_DMA、ZONE\_NORMAL、ZONE\_HIGHMEM
    \item \textbf{LRU 链表}:用于页面回收
\end{enumerate}

\subsection{2.6 内核:联合体的引入}

Linux 2.6 内核是一个里程碑式的版本,它引入了联合体(union)来复用 \texttt{struct page} 的空间:

\begin{lstlisting}[caption={Linux 2.6 早期的 struct page}]
/*
 * Linux 2.6.x 的 struct page
 */
struct page {
    unsigned long flags;
    atomic_t _count;
    atomic_t _mapcount;
    
    union {
        struct {
            /* 页缓存使用 */
            unsigned long private;
            struct address_space *mapping;
        };
        struct {
            /* slab 使用 */
            struct kmem_cache *slab_cache;
            void *freelist;
        };
    };
    
    pgoff_t index;
    struct list_head lru;
    
    /*
     * 在 64 位系统上,我们有更多空间
     */
#if defined(CONFIG_64BIT)
    /* 额外字段 */
#endif
};
\end{lstlisting}

联合体的引入是一把双刃剑:
\begin{itemize}
    \item \textbf{优点}:节省内存,一个页的元数据不需要为所有可能的用途分配空间
    \item \textbf{缺点}:代码可读性下降,容易出错,难以扩展
\end{itemize}

\subsection{复合页的引入}

Linux 2.6.16 左右引入了复合页(compound pages)的概念,这是为了支持巨页(huge pages)而做的重要改变:

\begin{lstlisting}[caption={复合页的实现}]
/*
 * 复合页的实现
 * 
 * 复合页由 2^order 个连续的基础页组成
 * 第一个页是"头页"(head page),其余是"尾页"(tail pages)
 */

/* 创建复合页 */
static void prep_compound_page(struct page *page, unsigned int order)
{
    int i;
    int nr_pages = 1 << order;
    
    /* 设置头页 */
    __SetPageHead(page);
    page[1].compound_order = order;
    page[1].compound_dtor = COMPOUND_PAGE_DTOR;
    
    /* 设置尾页 */
    for (i = 1; i < nr_pages; i++) {
        struct page *p = page + i;
        
        /*
         * compound_head 字段存储头页地址
         * 最低位设为 1 表示这是尾页
         */
        p->compound_head = (unsigned long)page + 1;
    }
    
    atomic_set(&page->_compound_mapcount, -1);
}

/* 获取复合页的头页 */
static inline struct page *compound_head(struct page *page)
{
    unsigned long head = READ_ONCE(page->compound_head);

    if (unlikely(head & 1))
        return (struct page *)(head - 1);
    return page;
}

/* 获取复合页的阶数 */
static inline unsigned int compound_order(struct page *page)
{
    if (!PageHead(page))
        return 0;
    return page[1].compound_order;
}
\end{lstlisting}

先看 \texttt{prep\_compound\_page()}。它把连续的 $2^{order}$ 个页组装成一个复合页。头页标上 \texttt{PG\_head},阶数和析构函数索引塞进 \texttt{page[1]}——为什么不放在头页自己身上?因为头页的联合体空间已经被别的字段占满了,只能借用第一个尾页来存。每个尾页的 \texttt{compound\_head} 字段存的是头页地址加 1,这个 ``+1'' 不是偏移一个 \texttt{struct page},而是在指针值的最低位设了个 1。\texttt{struct page} 天然对齐,地址最低位一定是 0,所以拿这一位当标记完全安全。

\texttt{compound\_head()} 大概是内核里调用频率最高的函数之一了。逻辑很简单:读出 \texttt{compound\_head} 字段,最低位是 1 就说明当前是尾页,去掉标记位就拿到了头页指针;否则自己就是头页。\texttt{unlikely()} 告诉编译器大多数情况下我们拿到的就是头页,让分支预测偏向快速路径。

再看 \texttt{compound\_order()},这里有个设计上让人不太舒服的地方:你必须先确认自己拿到的是头页,然后通过 \texttt{page[1]} 去第一个尾页里取阶数。元数据散落在不同的 \texttt{struct page} 里,后面维护起来处处是坑。

复合页的引入带来了显著的复杂性增加:
\begin{enumerate}
    \item 每个处理页的函数都需要考虑它可能是复合页的一部分
    \item 某些操作应该作用于整个复合页,某些应该作用于单个基础页
    \item 尾页的某些字段被复用来存储指向头页的指针
\end{enumerate}

\subsection{透明大页(THP)的影响}

Linux 2.6.38 引入了透明大页(Transparent Huge Pages)功能,它允许内核自动将连续的小页合并为大页:

\begin{lstlisting}[caption={透明大页的处理}]
/*
 * 透明大页相关代码
 * mm/huge_memory.c
 */

/* 尝试将页升级为大页 */
static int collapse_huge_page(struct mm_struct *mm,
                              unsigned long address,
                              struct page **hpage)
{
    struct vm_area_struct *vma;
    struct page *new_page;
    pmd_t *pmd, _pmd;
    int result = SCAN_FAIL;
    
    /* 分配一个大页 */
    new_page = alloc_pages(gfp_mask, HPAGE_PMD_ORDER);
    if (!new_page) {
        result = SCAN_ALLOC_HUGE_PAGE_FAIL;
        goto out;
    }
    
    /*
     * 这里需要:
     * 1. 锁定所有小页
     * 2. 复制数据到大页
     * 3. 更新页表
     * 4. 释放小页
     */
    
    /* 将新页设置为复合页 */
    prep_compound_page(new_page, HPAGE_PMD_ORDER);
    prep_transhuge_page(new_page);
    
    /* ... 复杂的迁移逻辑 ... */
    
    return result;
}

/* 拆分透明大页 */
int split_huge_page_to_list(struct page *page, struct list_head *list)
{
    struct page *head = compound_head(page);
    struct deferred_split *ds_queue = get_deferred_split_queue(head);

    /*
     * 拆分大页需要:
     * 1. 确保没有进程正在使用
     * 2. 分解复合页结构
     * 3. 更新所有相关的页表
     * 4. 处理引用计数
     */

    /* ... 复杂的拆分逻辑 ... */
}
\end{lstlisting}

\texttt{collapse\_huge\_page()} 做的事情概念上很简单:分配一个 order-9(512 页 = 2MB)的连续物理页,组装成复合页,再把原来那些零散小页的数据搬进来。但被省略号藏起来的迁移逻辑才是真正的噩梦:你得锁住所有原始小页、逐页拷贝数据、把涉及的所有进程页表从 PTE 映射改成 PMD 映射、最后释放小页。任何一步被中断或者失败,都要能干净地回滚。

反过来拆分就更麻烦了。\texttt{split\_huge\_page\_to\_list()} 第一步是 \texttt{compound\_head(page)},因为传进来的可能是 512 个子页中的任意一个。为什么要有个延迟拆分队列(\texttt{deferred\_split\_queue})?因为拆分的开销太大了:把一个 PMD 映射拆成 512 个 PTE、给每个子页建立独立的引用计数和映射计数、重建 LRU 链表。内核能拖就拖,实在逼不得已才做。

透明大页使问题进一步复杂化,因为一个``页''可能在运行时从小页变成大页,或者反过来。

\section{struct page 的现代形态}

\subsection{Linux 5.x 的 struct page}

到了 Linux 5.x,\texttt{struct page} 已经演变成一个庞大而复杂的结构:

\begin{lstlisting}[caption={Linux 5.x 的完整 struct page 定义}]
/*
 * Linux 5.x 的 struct page
 * include/linux/mm_types.h
 * 
 * 这是一个简化版,实际定义更加复杂
 */
struct page {
    /* 第一个字:flags,必须在最开始 */
    unsigned long flags;
    
    /*
     * 五个字的联合体
     * 
     * 每种页类型使用不同的字段解释
     * 这是历史演变的结果,也是问题的根源
     */
    union {
        struct {
            /* 用于页缓存和匿名页的 LRU */
            struct list_head lru;
            /* 页缓存: address_space
             * 匿名页: anon_vma 指针(最低位设置)
             * 交换缓存: swap 区域 */
            struct address_space *mapping;
            /* 页缓存中的索引 */
            pgoff_t index;
            /* 由使用者定义的私有数据 */
            unsigned long private;
        };
        
        struct {
            /* 用于 slab/slub/slob 分配器 */
            union {
                struct list_head slab_list;
                struct {
                    struct page *next;
                    int pages;
                    int pobjects;
                };
            };
            struct kmem_cache *slab_cache;
            void *freelist;
            union {
                void *s_mem;
                unsigned long counters;
                struct {
                    unsigned inuse:16;
                    unsigned objects:15;
                    unsigned frozen:1;
                };
            };
        };
        
        struct {
            /* 复合页的尾页使用 */
            unsigned long compound_head;
        };
        
        struct {
            /* 复合页的第二个尾页使用 */
            unsigned long _compound_pad_1;
            atomic_t compound_mapcount;
            unsigned int compound_nr;
            unsigned char compound_dtor;
            unsigned char compound_order;
        };
        
        struct {
            /* ZONE_DEVICE 页使用 */
            unsigned long _zd_pad_1;
            struct dev_pagemap *pgmap;
            void *zone_device_data;
        };
        
        struct rcu_head rcu_head;
        
        /* 更多联合体成员... */
    };
    
    union {
        /*
         * 映射计数
         * -1 表示没有映射
         * 0 表示有一个映射
         * N 表示有 N+1 个映射
         */
        atomic_t _mapcount;
        
        /*
         * 如果设置了 PG_buddy,则使用此字段
         */
        unsigned int page_type;
    };
    
    /* 引用计数 */
    atomic_t _refcount;
    
#ifdef CONFIG_MEMCG
    /* 内存控制组数据 */
    unsigned long memcg_data;
#endif
    
    /*
     * 在某些架构上,我们在页结构末尾存储 KASAN 影子内存
     */
#if defined(CONFIG_KASAN)
    /* KASAN 标签 */
#endif
};
\end{lstlisting}

到这个阶段,\texttt{struct page} 已经膨胀成了一个"瑞士军刀"式的结构体。\texttt{flags} 必须放在开头,大量底层代码硬编码了这个偏移。真正的重灾区是后面那个五字联合体(64 位系统上 40 字节):页缓存往里塞了 \texttt{lru}、\texttt{mapping}、\texttt{index}、\texttt{private};slab 把同一片内存解读为 \texttt{slab\_list}、\texttt{slab\_cache}、\texttt{freelist} 和一组位域计数器;复合页尾页只用了一个 \texttt{compound\_head};第二个尾页又存了 \texttt{compound\_order} 之类的元数据;\texttt{ZONE\_DEVICE} 页还有自己的一套。五六种完全不同的用途叠在同一片内存上,你读错了字段,编译器不会吭一声,运行时只会拿到一堆垃圾数据。

下面那个小联合体把 \texttt{\_mapcount} 和 \texttt{page\_type} 复用了。普通页用前者追踪页表映射数,伙伴系统空闲页和页表页用后者做类型标识,反正两者不会同时有意义。开了 \texttt{CONFIG\_MEMCG} 的话还得再加 8 字节给 cgroup 记账用。

\subsection{struct page 的大小问题}

让我们分析 \texttt{struct page} 的实际大小:

\begin{lstlisting}[caption={计算 struct page 的大小}]
/*
 * 在 64 位系统上,struct page 的典型大小
 */

/* 核心字段 */
unsigned long flags;          /*  8 字节 */
/* 联合体(5 个字) */          /* 40 字节 */
atomic_t _mapcount;           /*  4 字节 */
atomic_t _refcount;           /*  4 字节 */

#ifdef CONFIG_MEMCG
unsigned long memcg_data;     /*  8 字节 */
#endif

/* 总计: 约 64 字节(带 memcg) */

/*
 * 对于 4GB 内存的系统:
 * - 页数 = 4GB / 4KB = 1,048,576 页
 * - mem_map 大小 = 1,048,576 * 64 = 64 MB
 * 
 * 对于 256GB 内存的系统:
 * - 页数 = 256GB / 4KB = 67,108,864 页
 * - mem_map 大小 = 67,108,864 * 64 = 4 GB
 * 
 * 这是巨大的内存开销!
 */
\end{lstlisting}

算一笔账就知道问题有多大了。64 位系统上一个 \texttt{struct page} 大约 64 字节。4GB 内存的机器需要 64MB 来存 \texttt{mem\_map},还能接受;但 256GB 的服务器呢?4GB 纯粹用来描述物理内存本身,占总内存的 1.6\%。而且这些 \texttt{struct page} 不能换出到磁盘,页分配器随时要用它们,必须常驻物理内存。

folio 在这方面有个间接好处:一组连续页合成 folio 后,只有头页需要存完整元数据,尾页的部分字段可以腾出来做别的事。

\subsection{页标志(Page Flags)的复杂性}

\texttt{struct page} 的 \texttt{flags} 字段包含大量标志位:

\begin{lstlisting}[caption={页标志定义}]
/*
 * include/linux/page-flags.h
 * 
 * 页标志占用 unsigned long 的各个位
 */
enum pageflags {
    PG_locked,      /* 0: 页被锁定,正在进行 I/O */
    PG_referenced,  /* 1: 页最近被访问 */
    PG_uptodate,    /* 2: 页内容有效 */
    PG_dirty,       /* 3: 页被修改,需要写回 */
    PG_lru,         /* 4: 页在 LRU 链表上 */
    PG_active,      /* 5: 页在活跃 LRU 链表 */
    PG_workingset,  /* 6: 页在工作集中 */
    PG_waiters,     /* 7: 有进程在等待页解锁 */
    PG_error,       /* 8: I/O 错误 */
    PG_slab,        /* 9: slab 分配器使用 */
    PG_owner_priv_1,/* 10: 由所有者使用 */
    PG_arch_1,      /* 11: 架构特定 */
    PG_reserved,    /* 12: 保留页,不可换出 */
    PG_private,     /* 13: 有私有数据 */
    PG_private_2,   /* 14: 有更多私有数据 */
    PG_writeback,   /* 15: 正在写回 */
    PG_head,        /* 16: 复合页的头页 */
    PG_mappedtodisk,/* 17: 有磁盘块映射 */
    PG_reclaim,     /* 18: 即将被回收 */
    PG_swapbacked,  /* 19: 由交换空间支持 */
    PG_unevictable, /* 20: 不可驱逐 */
#ifdef CONFIG_MMU
    PG_mlocked,     /* 21: mlock() 锁定 */
#endif
#ifdef CONFIG_ARCH_USES_PG_UNCACHED
    PG_uncached,    /* 22: 未缓存 */
#endif
#ifdef CONFIG_MEMORY_FAILURE
    PG_hwpoison,    /* 23: 硬件内存错误 */
#endif
#if defined(CONFIG_PAGE_IDLE_FLAG)
    PG_young,       /* 24: 年轻页 */
    PG_idle,        /* 25: 空闲页 */
#endif
    /* ... 更多标志 ... */
    
    __NR_PAGEFLAGS,

    /* 复合页尾页的标志 */
    PG_head = PG_head,

    /* 不在 flags 中,但通过 page_type 判断 */
    PG_buddy,       /* 伙伴系统空闲页 */
    PG_offline,     /* 离线页 */
    PG_table,       /* 页表页 */
    PG_guard,       /* 守卫页 */
};
\end{lstlisting}

光看这个枚举,就能感受到 \texttt{struct page} 被迫承担了多少不相干的职责。25 个以上的标志位,涵盖 I/O 状态、LRU 管理、页缓存属性、内存类型标识,还有一堆受 \texttt{CONFIG\_xxx} 条件编译控制的可选标志。同一个标志位编号在不同内核配置下可能代表不同的东西。

还有一个容易踩坑的地方:\texttt{\_\_NR\_PAGEFLAGS} 后面那些 \texttt{PG\_buddy}、\texttt{PG\_table} 之类的,虽然写在同一个枚举里,其实根本不存在 \texttt{flags} 字段中。它们是通过 \texttt{page\_type} 这个联合体成员来编码的。另外,复合页只有头页有 \texttt{PG\_head},尾页并没有对称的 \texttt{PG\_tail},判断尾页得靠 \texttt{compound\_head} 字段的最低位。这种不对称性在写代码时很容易忘掉。

每个标志都有对应的操作函数:

\begin{lstlisting}[caption={页标志操作函数}]
/*
 * 对于每个标志 PG_xxx,都定义了以下函数:
 * - PageXxx(page): 测试标志是否设置
 * - SetPageXxx(page): 设置标志
 * - ClearPageXxx(page): 清除标志
 * - __SetPageXxx(page): 非原子设置
 * - __ClearPageXxx(page): 非原子清除
 * - TestSetPageXxx(page): 原子测试并设置
 * - TestClearPageXxx(page): 原子测试并清除
 */

/* 示例:PG_locked 的操作函数 */
static __always_inline int PageLocked(struct page *page)
{
    return test_bit(PG_locked, &page->flags);
}

static __always_inline void SetPageLocked(struct page *page)
{
    set_bit(PG_locked, &page->flags);
}

static __always_inline void ClearPageLocked(struct page *page)
{
    clear_bit(PG_locked, &page->flags);
}

static __always_inline int TestSetPageLocked(struct page *page)
{
    return test_and_set_bit(PG_locked, &page->flags);
}

/*
 * 问题:这些函数对复合页的行为是什么?
 * - PageLocked(tail_page) 会返回什么?
 * - SetPageLocked(tail_page) 会设置谁的标志?
 * 
 * 答案取决于标志的具体定义,非常容易出错!
 */
\end{lstlisting}

每个 \texttt{PG\_xxx} 标志都通过宏批量生成一族操作函数。带不带双下划线前缀的区别在于是否原子操作:\texttt{SetPageXxx()} 是原子的,多处理器安全;\texttt{\_\_SetPageXxx()} 不是,只在确定没有竞争时才能用,比如页刚分配出来的初始化阶段。\texttt{TestSetPageXxx()} 这种原子测试并设置的变体,常用来实现锁。

但这里藏着一个大坑。注意代码末尾的注释:这些函数直接操作你传入的那个 \texttt{page->flags},\textbf{不会}帮你调 \texttt{compound\_head()}。所以如果你拿一个尾页去调 \texttt{PageLocked()},查到的是尾页自己的 \texttt{PG\_locked} 位,而锁实际上设在头页上。结果?永远返回 false。这类 bug 排查起来极其痛苦,因为代码逻辑看上去完全没问题。

\section{具体问题案例分析}

\subsection{案例一:page\_address() 的困境}

\texttt{page\_address()} 函数用于获取页的虚拟地址,但它的行为在不同情况下有所不同:

\begin{lstlisting}[caption={page\_address() 的问题}]
/*
 * page_address() 的实现分析
 */

/* 在没有高端内存的系统上(大多数 64 位系统) */
static inline void *page_address(const struct page *page)
{
    return (void *)((unsigned long)page_to_pfn(page) << PAGE_SHIFT);
}

/*
 * 问题场景:
 */
void problematic_function(struct page *page)
{
    void *addr;
    
    /* 
     * 如果 page 是复合页的尾页会怎样?
     * page_address() 返回的是这个尾页的地址
     * 但调用者可能期望整个复合页的起始地址
     */
    addr = page_address(page);
    
    /* 
     * 现在 addr 指向复合页中间的某处
     * 如果我们尝试访问"整个页"的数据,会出问题
     */
    memset(addr, 0, PAGE_SIZE);  /* 这可能是错的! */
}

/* 更安全的做法 */
void safer_function(struct page *page)
{
    struct page *head = compound_head(page);
    void *addr = page_address(head);
    size_t size = page_size(head);  /* 获取复合页的完整大小 */
    
    memset(addr, 0, size);
}
\end{lstlisting}

\texttt{page\_address()} 不管你传的是不是尾页,它就老老实实返回那个页的地址。如果你传了一个复合页的第 3 个尾页进去,拿到的就是那个尾页对应的 4KB 区域的起始地址,跟复合页的起始地址差了 12KB。接下来 \texttt{memset(addr, 0, PAGE\_SIZE)} 只清零了 4KB,如果你本意是清整个复合页,剩下的数据纹丝不动。

正确的写法在 \texttt{safer\_function()} 里:先 \texttt{compound\_head()} 拿到头页,再取地址,用 \texttt{page\_size()} 拿到真实大小。问题是谁来保证所有调用者都记得这么做?没有人。编译器帮不了你,只能靠开发者自觉和代码审查。

\subsection{案例二:页锁的语义}

页锁(page lock)的行为对于复合页来说特别混乱:

\begin{lstlisting}[caption={页锁的问题}]
/*
 * 页锁操作分析
 */

/* lock_page() 的实现 */
static inline void lock_page(struct page *page)
{
    /* 注意:这里会自动找到头页 */
    page = compound_head(page);
    
    might_sleep();
    if (!trylock_page(page))
        __lock_page(page);
}

/* trylock_page() 的实现 */
static inline int trylock_page(struct page *page)
{
    /* 也是操作头页 */
    page = compound_head(page);
    return (likely(!test_and_set_bit_lock(PG_locked, &page->flags)));
}

/*
 * 问题:以下代码正确吗?
 */
void questionable_code(struct page *page)
{
    /* 传入了一个尾页 */
    lock_page(page);
    
    /* 现在我们锁定了整个复合页 */
    
    /* 但是,我们使用的是尾页的指针 */
    /* 某些操作可能仍然对尾页做错误的假设 */
    
    /* 比如这个检查 */
    if (PageLocked(page)) {  /* 这会检查头页还是尾页? */
        /* ... */
    }
    
    unlock_page(page);  /* 这会解锁头页还是尾页? */
}

/*
 * PageLocked() 的实际实现
 */
static inline int PageLocked(struct page *page)
{
    /* 注意:这里 *没有* 自动调用 compound_head() */
    return test_bit(PG_locked, &page->flags);
}

/*
 * 所以 PageLocked(tail_page) 总是返回 false!
 * 因为锁是在头页上设置的!
 * 
 * 这是一个极其容易出错的 API 设计
 */
\end{lstlisting}

这是整个 \texttt{struct page} API 里最阴险的不一致。\texttt{lock\_page()} 内部会调 \texttt{compound\_head()},传尾页进去也能正确锁住头页。\texttt{unlock\_page()} 同理。但 \texttt{PageLocked()} 不会。它直接查传入页的 \texttt{PG\_locked} 位。

所以 \texttt{questionable\_code()} 里发生了什么?\texttt{lock\_page(page)} 锁住了头页,没问题。然后 \texttt{PageLocked(page)} 去查尾页的标志位,尾页上当然没设过 \texttt{PG\_locked},永远返回 false。你以为锁了,检查却告诉你没锁。\texttt{unlock\_page()} 倒是能正确解锁,因为它也会自动找头页。

哪些函数自动处理复合页,哪些不处理?没有规律可循,你只能逐个翻源码确认。这就是 folio 要用类型系统解决的核心痛点。

\subsection{案例三:引用计数的混乱}

复合页的引用计数处理更加复杂:

\begin{lstlisting}[caption={引用计数的问题}]
/*
 * 复合页的引用计数分析
 */

/* get_page() 的实现 */
static inline void get_page(struct page *page)
{
    page = compound_head(page);
    
    /* 增加头页的引用计数 */
    VM_BUG_ON_PAGE(page_ref_zero_or_close_to_overflow(page), page);
    page_ref_inc(page);
}

/* put_page() 的实现 */
void put_page(struct page *page)
{
    page = compound_head(page);
    
    if (put_page_testzero(page))
        __put_page(page);
}

/*
 * 但是,某些代码需要对单个子页进行引用计数
 * 比如 GUP(Get User Pages)的 FOLL_GET 行为
 */

/*
 * 这导致了两套引用计数系统:
 * 1. page->_refcount: 整个复合页的引用计数
 * 2. 每个子页的 mapcount: 映射到页表的计数
 * 
 * 两者的语义不同,容易混淆
 */

/* 检查页是否被引用 */
int page_count(struct page *page)
{
    return atomic_read(&compound_head(page)->_refcount);
}

/* 检查页是否被映射 */
int page_mapped(struct page *page)
{
    int mapcount;
    
    page = compound_head(page);
    
    /* 复合页有两个 mapcount */
    if (PageCompound(page)) {
        /* compound_mapcount: 整个复合页的映射 */
        mapcount = atomic_read(&page->_compound_mapcount) + 1;
        /* 加上各个子页的映射 */
        mapcount += atomic_read(&page->_mapcount) + 1;
    } else {
        mapcount = atomic_read(&page->_mapcount) + 1;
    }
    
    return mapcount > 0;
}
\end{lstlisting}

\texttt{get\_page()} 和 \texttt{put\_page()} 还好理解,都是先 \texttt{compound\_head()} 找到头页,再操作头页的 \texttt{\_refcount}。引用归零时 \texttt{\_\_put\_page()} 负责释放整个复合页。

麻烦的是 \texttt{page\_mapped()}。复合页其实有两套映射计数:\texttt{\_compound\_mapcount} 记录整个复合页被当作大页映射了几次(比如 PMD 映射 2MB),每个子页的 \texttt{\_mapcount} 记录该子页被独立映射了几次(比如 fork 之后 COW)。两者得加起来才是对的。拆分大页时还得把这两套计数器拆开重分配——搞错了不是页面泄漏就是 use-after-free。

\subsection{案例四:slab 与页缓存的冲突}

\texttt{struct page} 试图同时服务于 slab 分配器和页缓存,这导致了字段复用的问题:

\begin{lstlisting}[caption={slab 与页缓存的字段冲突}]
/*
 * struct page 的联合体问题
 */
struct page {
    union {
        struct {
            /* 页缓存使用 */
            struct list_head lru;
            struct address_space *mapping;
            pgoff_t index;
            unsigned long private;
        };
        struct {
            /* slab 使用 */
            struct list_head slab_list;
            struct kmem_cache *slab_cache;
            void *freelist;
            unsigned int inuse;
        };
    };
};

/*
 * 问题:如何判断一个页是 slab 页还是页缓存页?
 */
static inline bool page_is_slab(struct page *page)
{
    return PageSlab(page);  /* 检查 PG_slab 标志 */
}

/*
 * 但是在某些代码路径中,我们可能还没有设置 PG_slab
 * 或者刚刚清除了它,这时候字段的解释就变得不确定
 */

/*
 * 更糟糕的是,某些调试功能(如 KASAN)需要在两种情况下
 * 都能正确工作,这增加了复杂性
 */

/* 
 * KASAN 需要知道内存的类型来进行正确的检查
 * 但 struct page 的类型是动态变化的
 */
void kasan_check_page(struct page *page)
{
    if (PageSlab(page)) {
        /* 作为 slab 页处理 */
        kasan_check_slab(page);
    } else if (page->mapping) {
        /* 作为页缓存页处理 */
        kasan_check_pagecache(page);
    } else {
        /* ??? 不确定的状态 */
    }
}
\end{lstlisting}

页缓存用 \texttt{lru}、\texttt{mapping}、\texttt{index}、\texttt{private},slab 把同一片内存解读成 \texttt{slab\_list}、\texttt{slab\_cache}、\texttt{freelist}、\texttt{inuse}。两组字段叠在一起,靠 \texttt{PG\_slab} 标志位来区分。

看 \texttt{kasan\_check\_page()} 就知道这有多别扭:KASAN 要根据页类型选检查策略,但页从 slab 释放回伙伴系统的过程中,\texttt{PG\_slab} 可能已经清了,\texttt{mapping} 还没初始化。这时候你查哪个字段都不靠谱。这种"半生不熟"的中间状态也是后来 \texttt{struct slab} 被独立出去的原因之一。

\section{社区的讨论与争议}

\subsection{早期的改进尝试}

在 folio 之前,社区曾尝试过其他方式来解决 \texttt{struct page} 的问题:

\begin{enumerate}
    \item \textbf{compound\_page() 自动处理}
    
    一些函数开始在内部自动调用 \texttt{compound\_head()},但这不是一致的行为。
    
    \item \textbf{thp\_xxx() 函数族}
    
    为透明大页创建专门的函数,如 \texttt{thp\_lock()}、\texttt{thp\_unlock()}。但这只是部分解决问题。
    
    \item \textbf{文档和注释}
    
    试图通过更好的文档来解释函数的预期行为。但在实践中,开发者很少仔细阅读这些文档。
\end{enumerate}

\subsection{folio 提案的初始反应}

2020-2021 年,当 Matthew Wilcox 首次提出 folio 时,社区的反应是混合的:

\textbf{支持的观点:}
\begin{itemize}
    \item 提供了急需的类型安全
    \item 简化了复合页的处理
    \item 为未来的改进奠定基础
    \item 使代码意图更加清晰
\end{itemize}

\textbf{反对/担忧的观点:}
\begin{itemize}
    \item 大规模的 API 变更会引入 bug
    \item 需要长时间的迁移期
    \item 又一个需要学习的新概念
    \item 可能影响性能
\end{itemize}

\subsection{Linus Torvalds 的观点}

Linus Torvalds 在邮件列表中表达了对 folio 的总体支持,同时也指出了一些需要注意的问题:

\begin{quote}
``I like the direction, and I think the folio concept is a good one. The fact that it makes the code clearer about when it's dealing with a single page vs a compound page is a definite improvement.

However, I want to make sure we don't break things. The MM code is complex enough as it is, and we need to be careful about the transition.''

(我喜欢这个方向,folio 的概念很好。它使代码在处理单页和复合页时更加清晰,这绝对是一个改进。

但是,我要确保我们不会破坏任何东西。内存管理代码已经够复杂了,我们需要谨慎地进行过渡。)
\end{quote}

\section{struct page 的未来}

\subsection{分离策略}

内核社区采取了一种渐进式的策略来重构 \texttt{struct page}:

\begin{enumerate}
    \item \textbf{引入 folio}:用于页缓存和文件系统
    \item \textbf{引入 struct slab}:将 slab 相关字段分离
    \item \textbf{保留 struct page}:作为最底层的页框描述符
\end{enumerate}

\begin{lstlisting}[caption={struct slab 的分离}]
/*
 * struct slab: 从 struct page 分离的 slab 描述符
 * 这是正在进行的工作(Linux 5.17+)
 */
struct slab {
    unsigned long flags;
    
    union {
        struct list_head slab_list;
        struct {
            struct slab *next;
            int slabs;
        };
    };
    
    struct kmem_cache *slab_cache;
    void *freelist;
    
    union {
        unsigned long counters;
        struct {
            unsigned inuse:16;
            unsigned objects:15;
            unsigned frozen:1;
        };
    };
    
    atomic_t __page_refcount;
};

/*
 * slab 与 page 的转换
 */
static inline struct slab *page_slab(struct page *page)
{
    VM_BUG_ON_PAGE(!PageSlab(page), page);
    return (struct slab *)page;
}

static inline struct page *slab_page(struct slab *slab)
{
    return (struct page *)slab;
}
\end{lstlisting}

思路跟 folio 一样:新结构体和 \texttt{struct page} 内存布局兼容,直接强转就行。原来挤在联合体里的字段现在各有各的名字,读起来终于像正常代码了。注意 \texttt{page\_slab()} 里的 \texttt{VM\_BUG\_ON\_PAGE(!PageSlab(page), page)}——传了个非 slab 页进来?直接 BUG。比起联合体方案"悄悄读错字段还不报错",进步不是一点半点。

\subsection{长期愿景}

长期来看,\texttt{struct page} 可能会变成一个更简单的结构,主要包含:

\begin{itemize}
    \item 页框号(PFN)相关信息
    \item 基本的引用计数
    \item 类型标识
\end{itemize}

而具体的功能则由专门的结构体承担:
\begin{itemize}
    \item \texttt{struct folio}:页缓存、文件映射
    \item \texttt{struct slab}:slab 分配器
    \item 其他潜在的专用结构
\end{itemize}

\section{小结}

本章详细分析了 \texttt{struct page} 的历史演变和困境:

\begin{enumerate}
    \item \textbf{历史演变}:从简单的字节数组到复杂的联合体结构
    
    \item \textbf{复合页的引入}:为大页支持带来的复杂性
    
    \item \textbf{具体问题}:
    \begin{itemize}
        \item API 语义不一致
        \item 类型安全缺失
        \item 引用计数混乱
        \item 字段复用冲突
    \end{itemize}
    
    \item \textbf{社区讨论}:folio 提案的支持与争议
    
    \item \textbf{未来方向}:通过引入 folio、slab 等专用结构来简化 \texttt{struct page}
\end{enumerate}

理解这些历史背景和问题,有助于我们更好地理解 folio 的设计决策。在下一章中,我们将深入探讨 folio 的设计哲学。



\end{document}
