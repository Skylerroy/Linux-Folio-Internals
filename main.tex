% 深入理解 Linux Folio:从设计思想到代码实现
% 主文件 main.tex

\documentclass[11pt,a4paper,openany]{book}

% 中文支持
\usepackage[UTF8]{ctex}
\usepackage{fontspec}

% 页面设置
\usepackage[top=2.5cm, bottom=2.5cm, left=3cm, right=3cm]{geometry}

% 代码高亮
\usepackage{listings}
\usepackage{xcolor}

% 定义代码样式
\lstdefinestyle{cstyle}{
    language=C,
    basicstyle=\ttfamily\small,
    keywordstyle=\color{blue}\bfseries,
    commentstyle=\color{gray}\itshape,
    stringstyle=\color{red},
    numbers=left,
    numberstyle=\tiny\color{gray},
    stepnumber=1,
    numbersep=8pt,
    backgroundcolor=\color{gray!5},
    showspaces=false,
    showstringspaces=false,
    showtabs=false,
    frame=single,
    rulecolor=\color{gray!30},
    tabsize=4,
    captionpos=b,
    breaklines=true,
    breakatwhitespace=false,
    escapeinside={(*@}{@*)},
    morekeywords={bool, size_t, sector_t, pgoff_t, loff_t, u8, u16, u32, u64, s8, s16, s32, s64, 
                  atomic_t, atomic_long_t, spinlock_t, rwlock_t, mutex, 
                  gfp_t, fgf_t, vm_fault_t,
                  __always_inline, __must_check, __init, __exit,
                  likely, unlikely, container_of,
                  EXPORT_SYMBOL, MODULE_LICENSE, MODULE_AUTHOR,
                  BUG_ON, WARN_ON, WARN_ON_ONCE, BUILD_BUG_ON,
                  folio, page, address_space, file, inode, super_block,
                  bio, bio_vec, request, request_queue}
}

\lstset{style=cstyle}

% 图表
\usepackage{graphicx}
\usepackage{float}
\usepackage{tikz}
\usetikzlibrary{shapes,arrows,positioning,calc,fit,backgrounds}

% 表格
\usepackage{booktabs}
\usepackage{longtable}
\usepackage{multirow}
\usepackage{array}
\usepackage{tabularx}

% 超链接
\usepackage{hyperref}
\hypersetup{
    colorlinks=true,
    linkcolor=blue,
    filecolor=magenta,
    urlcolor=cyan,
    citecolor=green,
    pdftitle={深入理解 Linux Folio},
    pdfauthor={},
    bookmarks=true,
}

% 目录深度
\setcounter{tocdepth}{3}
\setcounter{secnumdepth}{3}

% 章节样式
\usepackage{titlesec}
\titleformat{\chapter}[display]
    {\normalfont\huge\bfseries}
    {\chaptertitlename\ \thechapter}
    {20pt}
    {\Huge}

% 页眉页脚
\usepackage{fancyhdr}
\pagestyle{fancy}
\fancyhf{}
\fancyhead[LE,RO]{\thepage}
\fancyhead[RE]{\leftmark}
\fancyhead[LO]{\rightmark}
\renewcommand{\headrulewidth}{0.4pt}

% 引用
\usepackage{epigraph}

% 脚注
\usepackage{footnote}

% 索引
\usepackage{makeidx}
\makeindex

% 附录
\usepackage{appendix}

% 算法
\usepackage{algorithm}
\usepackage{algorithmic}

% 数学
\usepackage{amsmath}
\usepackage{amssymb}

% 分章节编译
\usepackage{subfiles}

% 自定义函数文档环境
\newenvironment{functiondesc}[2]{%
    \paragraph{\texttt{#1}(#2)}
    \mbox{}\\
}{%
    \par\medskip
}

% 函数参数和返回值命令
\newcommand{\param}[2]{\noindent\textbf{参数 #1:} #2\par}
\newcommand{\returns}[1]{\noindent\textbf{返回值:} #1\par}

% 自定义框架环境
\usepackage{tcolorbox}
\tcbuselibrary{skins,breakable}

\newtcolorbox{warningbox}{
    colback=yellow!10,
    colframe=orange!75!black,
    title={\textbf{警告}},
    fonttitle=\bfseries,
    breakable
}

\newtcolorbox{keypoints}[1]{
    colback=blue!5,
    colframe=blue!75!black,
    title={#1},
    fonttitle=\bfseries,
    breakable
}

\newtcolorbox{furtherreading}[1]{
    colback=green!5,
    colframe=green!75!black,
    title={#1},
    fonttitle=\bfseries,
    breakable
}

\newtcolorbox{checklistbox}[1]{
    colback=gray!5,
    colframe=gray!75!black,
    title={#1},
    fonttitle=\bfseries,
    breakable
}

%============================================================
% 文档开始
%============================================================

\begin{document}

%------------------------------------------------------------
% 封面
%------------------------------------------------------------
\begin{titlepage}
    \centering
    \vspace*{3cm}
    
    {\Huge\bfseries 深入理解 Linux Folio}\\[0.5cm]
    {\Large\itshape 从设计思想到代码实现}\\[2cm]
    
    {\large 基于 Linux 5.16 -- 6.x 内核}\\[3cm]
    
    \vfill
    
    {\large 2026年2月}
\end{titlepage}

%------------------------------------------------------------
% 版权页
%------------------------------------------------------------
\newpage
\thispagestyle{empty}
\vspace*{\fill}
\begin{center}
    \textbf{深入理解 Linux Folio:从设计思想到代码实现}\\[1cm]
    内核版本:Linux 5.16 -- 6.x\\[0.5cm]
    本书基于 GPLv2 许可证发布\\[2cm]
    
    \textit{``The folio is the new unit of memory management in the Linux kernel.''}\\
    \textit{--- Matthew Wilcox}
\end{center}
\vspace*{\fill}

%------------------------------------------------------------
% 前言
%------------------------------------------------------------
\chapter*{前言}
\addcontentsline{toc}{chapter}{前言}

Linux 内核的内存管理子系统是整个操作系统最复杂、最核心的部分之一。在过去的三十年里,\texttt{struct page} 一直是内核表示物理内存页的基础数据结构。然而,随着内核的不断演进,\texttt{struct page} 逐渐暴露出诸多设计上的问题,特别是在处理复合页(compound page)和大页(huge page)时的复杂性。

2021年,Matthew Wilcox 提出了 \textbf{folio} 的概念,并在 Linux 5.16 中首次合并。folio 不是对 \texttt{struct page} 的简单替换,而是一种全新的抽象——它代表一个或多个物理连续页的集合,为内核的内存管理提供了更清晰、更高效的接口。

本书旨在全面、深入地介绍 folio 的设计思想、实现细节和最佳实践。我们将从以下几个方面展开:

\begin{itemize}
    \item \textbf{历史与动机}:为什么需要 folio?\texttt{struct page} 存在哪些问题?
    \item \textbf{设计哲学}:folio 的核心设计思想和设计决策
    \item \textbf{数据结构}:folio 的内部结构和关键字段
    \item \textbf{API 详解}:完整的 folio API 及其使用方法
    \item \textbf{子系统集成}:folio 与页缓存、文件系统、内存回收的交互
    \item \textbf{演进历史}:从 5.16 到 6.x 的发展过程
    \item \textbf{实践指南}:如何将现有代码迁移到 folio
\end{itemize}

\subsection*{读者对象}

本书面向以下读者:

\begin{itemize}
    \item Linux 内核开发者,特别是从事内存管理、文件系统、块设备驱动开发的工程师
    \item 希望深入理解 Linux 内核内存管理机制的系统程序员
    \item 对操作系统设计感兴趣的研究人员和学生
\end{itemize}

阅读本书需要具备以下基础:

\begin{itemize}
    \item 熟悉 C 语言编程
    \item 了解基本的操作系统概念(虚拟内存、页表、进程等)
    \item 有一定的 Linux 内核开发经验(或至少阅读过内核代码)
\end{itemize}

\subsection*{代码版本}

本书的代码示例基于以下内核版本:

\begin{itemize}
    \item Linux 5.16:folio 首次引入
    \item Linux 5.18:folio 在页缓存中的广泛应用
    \item Linux 6.0:folio API 的稳定化
    \item Linux 6.x:当前的成熟实现
\end{itemize}

我们会在相关章节明确标注代码所属的内核版本,并讨论不同版本之间的差异。

\subsection*{致谢}

感谢 Matthew Wilcox 对 folio 的杰出贡献,以及所有参与 folio 开发和审阅的内核开发者。感谢 LWN.net 提供的高质量技术文章,它们是理解 folio 演进的重要参考资料。

\vspace{1cm}
\hfill 作者

\hfill 2026年2月

%------------------------------------------------------------
% 目录
%------------------------------------------------------------
\tableofcontents

%------------------------------------------------------------
% 正文
%------------------------------------------------------------

% 第1章
\subfile{chapters/chapter01}

% 第2章
\subfile{chapters/chapter02}

% 第3章
\subfile{chapters/chapter03}

% 第4章
\subfile{chapters/chapter04}

% 第5章
\subfile{chapters/chapter05}

% 第6章
\subfile{chapters/chapter06}

% 第7章
\subfile{chapters/chapter07}

% 第8章
\subfile{chapters/chapter08}

%------------------------------------------------------------
% 附录
%------------------------------------------------------------
\appendix

\subfile{appendix/appendix_a}
\subfile{appendix/appendix_b}
\subfile{appendix/appendix_c}

%------------------------------------------------------------
% 参考文献
%------------------------------------------------------------
\chapter*{参考文献}
\addcontentsline{toc}{chapter}{参考文献}

\begin{enumerate}
    \item Wilcox, Matthew. "Folios." LWN.net, March 2021.
    \item Wilcox, Matthew. "Pulling slabs out of struct page." LWN.net, 2021.
    \item Corbet, Jonathan. "An introduction to compound pages." LWN.net, 2007.
    \item Linux Kernel Documentation. "Memory Management." \url{https://www.kernel.org/doc/html/latest/mm/}
    \item Linux Kernel Source Code. \url{https://elixir.bootlin.com/linux/latest/source}
    \item Gorman, Mel. "Understanding the Linux Virtual Memory Manager." 2004.
    \item Love, Robert. "Linux Kernel Development." 3rd Edition, 2010.
    \item Bovet, Daniel P. and Cesati, Marco. "Understanding the Linux Kernel." 3rd Edition, 2005.
\end{enumerate}

%------------------------------------------------------------
% 索引
%------------------------------------------------------------
\printindex

\end{document}