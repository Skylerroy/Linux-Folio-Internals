\documentclass[../main.tex]{subfiles}
\begin{document}
\chapter{folio API参考手册}
\label{appendix:api-reference}

\section{核心folio操作函数}

\subsection{folio分配和释放}

\begin{functiondesc}{alloc\_folio}{gfp\_t gfp, unsigned int order}
分配一个新的folio对象。

\param{gfp}{内存分配标志,如GFP\_KERNEL, GFP\_ATOMIC等}
\param{order}{页阶数,0表示单页,1表示2页,2表示4页等}
\returns{成功时返回folio指针,失败时返回NULL}

\begin{lstlisting}[language=C]
struct folio *folio = alloc_folio(GFP_KERNEL, 2);  /* 分配4页的folio */
if (!folio) {
    /* 处理分配失败 */
    return -ENOMEM;
}
\end{lstlisting}
\end{functiondesc}

\begin{functiondesc}{folio\_get}{struct folio *folio}
增加folio的引用计数。

\param{folio}{要增加引用计数的folio}
\end{functiondesc}

\begin{functiondesc}{folio\_put}{struct folio *folio}
减少folio的引用计数,当引用计数归零时释放folio。

\param{folio}{要减少引用计数的folio}
\end{functiondesc}

\subsection{folio状态查询和设置}

\begin{functiondesc}{folio\_test\_uptodate}{struct folio *folio}
检查folio是否标记为最新的。

\param{folio}{要检查的folio}
\returns{如果是最新状态返回true,否则返回false}
\end{functiondesc}

\begin{functiondesc}{folio\_set\_uptodate}{struct folio *folio}
将folio标记为最新的。

\param{folio}{要设置的folio}
\end{functiondesc}

\begin{functiondesc}{folio\_clear\_uptodate}{struct folio *folio}
清除folio的最新标记。

\param{folio}{要清除标记的folio}
\end{functiondesc}

\begin{functiondesc}{folio\_test\_dirty}{struct folio *folio}
检查folio是否为脏页。

\param{folio}{要检查的folio}
\returns{如果是脏页返回true,否则返回false}
\end{functiondesc}

\begin{functiondesc}{folio\_mark\_dirty}{struct folio *folio}
将folio标记为脏页。

\param{folio}{要标记的folio}
\end{functiondesc}

\begin{functiondesc}{folio\_clear\_dirty}{struct folio *folio}
清除folio的脏页标记。

\param{folio}{要清除标记的folio}
\end{functiondesc}

\subsection{folio锁定操作}

\begin{functiondesc}{folio\_lock}{struct folio *folio}
锁定folio,防止并发访问。

\param{folio}{要锁定的folio}
\end{functiondesc}

\begin{functiondesc}{folio\_unlock}{struct folio *folio}
解锁folio。

\param{folio}{要解锁的folio}
\end{functiondesc}

\begin{functiondesc}{folio\_trylock}{struct folio *folio}
尝试锁定folio,不会阻塞。

\param{folio}{要尝试锁定的folio}
\returns{成功锁定返回true,否则返回false}
\end{functiondesc}

\begin{functiondesc}{folio\_lock\_killable}{struct folio *folio}
可被信号中断的folio锁定。

\param{folio}{要锁定的folio}
\returns{成功返回0,被信号中断返回-EINTR}
\end{functiondesc}

\subsection{folio数据访问}

\begin{functiondesc}{folio\_address}{struct folio *folio}
获取folio的内核虚拟地址。

\param{folio}{要获取地址的folio}
\returns{folio的内核虚拟地址}
\end{functiondesc}

\begin{functiondesc}{kmap\_local\_folio}{struct folio *folio, unsigned long offset}
映射folio到本地CPU的内核地址空间。

\param{folio}{要映射的folio}
\param{offset}{folio内的偏移量}
\returns{映射后的内核地址}
\end{functiondesc}

\begin{functiondesc}{kunmap\_local}{void *addr}
取消本地folio映射。

\param{addr}{之前映射返回的地址}
\end{functiondesc}

\begin{functiondesc}{folio\_copy\_to\_iter}{struct folio *folio, size\_t offset, size\_t bytes, struct iov\_iter *i}
将folio数据拷贝到iov\_iter。

\param{folio}{源folio}
\param{offset}{folio内起始偏移}
\param{bytes}{要拷贝的字节数}
\param{i}{目标iov\_iter}
\returns{实际拷贝的字节数}
\end{functiondesc}

\begin{functiondesc}{folio\_copy\_from\_iter}{struct folio *folio, size\_t offset, size\_t bytes, struct iov\_iter *i}
从iov\_iter拷贝数据到folio。

\param{folio}{目标folio}
\param{offset}{folio内起始偏移}
\param{bytes}{要拷贝的字节数}
\param{i}{源iov\_iter}
\returns{实际拷贝的字节数}
\end{functiondesc}

\section{复合folio操作}

\subsection{复合folio属性查询}

\begin{functiondesc}{folio\_test\_compound}{struct folio *folio}
检查folio是否为复合folio。

\param{folio}{要检查的folio}
\returns{如果是复合folio返回true,否则返回false}
\end{functiondesc}

\begin{functiondesc}{folio\_order}{struct folio *folio}
获取folio的阶数。

\param{folio}{要查询的folio}
\returns{folio的阶数}
\end{functiondesc}

\begin{functiondesc}{folio\_nr\_pages}{struct folio *folio}
获取folio包含的页数。

\param{folio}{要查询的folio}
\returns{folio包含的页数}
\end{functiondesc}

\begin{functiondesc}{folio\_size}{struct folio *folio}
获取folio的总大小(字节数)。

\param{folio}{要查询的folio}
\returns{folio的总大小}
\end{functiondesc}

\subsection{复合folio遍历}

\begin{functiondesc}{folio\_first\_page}{struct folio *folio}
获取folio的第一个页。

\param{folio}{要查询的folio}
\returns{folio的第一个页}
\end{functiondesc}

\begin{functiondesc}{folio\_last\_page}{struct folio *folio}
获取folio的最后一个页。

\param{folio}{要查询的folio}
\returns{folio的最后一个页}
\end{functiondesc}

\begin{functiondesc}{folio\_for\_each\_page}{page, folio}
遍历folio中的所有页。

\param{page}{循环变量}
\param{folio}{要遍历的folio}

\begin{lstlisting}[language=C]
struct page *page;
struct folio *folio = get_some_folio();

folio_for_each_page(page, folio) {
    /* 处理每个页 */
    process_page(page);
}
\end{lstlisting}
\end{functiondesc}

\section{页缓存操作}

\subsection{页缓存查找和插入}

\begin{functiondesc}{filemap\_get\_folio}{struct address\_space *mapping, pgoff\_t index}
在页缓存中查找指定索引的folio。

\param{mapping}{地址空间}
\param{index}{页索引}
\returns{找到的folio指针,未找到返回NULL}
\end{functiondesc}

\begin{functiondesc}{filemap\_insert\_folio}{struct address\_space *mapping, struct folio *folio, pgoff\_t index}
将folio插入到页缓存。

\param{mapping}{地址空间}
\param{folio}{要插入的folio}
\param{index}{页索引}
\returns{成功返回0,失败返回错误码}
\end{functiondesc}

\begin{functiondesc}{filemap\_remove\_folio}{struct folio *folio}
从页缓存中移除folio。

\param{folio}{要移除的folio}
\end{functiondesc}

\subsection{页缓存管理}

\begin{functiondesc}{folio\_add\_lru}{struct folio *folio}
将folio添加到LRU链表。

\param{folio}{要添加的folio}
\end{functiondesc}

\begin{functiondesc}{folio\_del\_lru}{struct folio *folio}
将folio从LRU链表中删除。

\param{folio}{要删除的folio}
\end{functiondesc}

\begin{functiondesc}{folio\_activate}{struct folio *folio}
激活folio(移动到活跃LRU链表)。

\param{folio}{要激活的folio}
\end{functiondesc}

\begin{functiondesc}{folio\_deactivate}{struct folio *folio}
停用folio(移动到非活跃LRU链表)。

\param{folio}{要停用的folio}
\end{functiondesc}

\section{类型转换函数}

\subsection{folio与page转换}

\begin{functiondesc}{page\_folio}{struct page *page}
将page指针转换为folio指针。

\param{page}{要转换的page}
\returns{对应的folio指针}
\end{functiondesc}

\begin{functiondesc}{folio\_page}{struct folio *folio, unsigned int n}
获取folio中的第n个页。

\param{folio}{源folio}
\param{n}{页索引(从0开始)}
\returns{第n个页的指针}
\end{functiondesc}

\begin{functiondesc}{page\_is\_folio\_head}{struct page *page}
检查page是否是folio的头页。

\param{page}{要检查的page}
\returns{如果是头页返回true,否则返回false}
\end{functiondesc}

\section{调试和诊断函数}

\subsection{folio验证}

\begin{functiondesc}{folio\_assert}{struct folio *folio}
验证folio的基本属性。

\param{folio}{要验证的folio}
\end{functiondesc}

\begin{functiondesc}{folio\_dump}{struct folio *folio}
转储folio的详细信息。

\param{folio}{要转储的folio}
\end{functiondesc}

\subsection{统计信息}

\begin{functiondesc}{folio\_stat\_incr}{enum folio\_stat\_item item, unsigned long count}
增加folio统计计数。

\param{item}{统计项目}
\param{count}{增加的数量}
\end{functiondesc}

\begin{functiondesc}{folio\_stat\_decr}{enum folio\_stat\_item item, unsigned long count}
减少folio统计计数。

\param{item}{统计项目}
\param{count}{减少的数量}
\end{functiondesc}

\section{错误处理}

\subsection{常见错误码}

\begin{table}[htbp]
\centering
\caption{folio操作常见错误码}
\label{tab:folio-error-codes}
\begin{tabular}{|l|l|p{6cm}|}
\hline
\textbf{错误码} & \textbf{数值} & \textbf{含义} \\
\hline
ENOMEM & -12 & 内存分配失败 \\
EINTR & -4 & 操作被信号中断 \\
EINVAL & -22 & 无效参数 \\
EIO & -5 & I/O错误 \\
EBUSY & -16 & 资源忙 \\
\hline
\end{tabular}
\end{table}

\subsection{错误处理最佳实践}

\begin{lstlisting}[language=C,caption={错误处理示例}]
static int safe_folio_operation(struct folio *folio)
{
    int ret;
    
    /* 参数验证 */
    if (!folio)
        return -EINVAL;
    
    /* 锁定folio */
    if (folio_lock_killable(folio))
        return -EINTR;
    
    /* 执行操作 */
    ret = do_folio_operation(folio);
    if (ret) {
        pr_err("Folio operation failed: %d\n", ret);
        goto unlock;
    }
    
    /* 标记folio状态 */
    folio_mark_dirty(folio);
    
unlock:
    folio_unlock(folio);
    return ret;
}
\end{lstlisting}


\end{document}
