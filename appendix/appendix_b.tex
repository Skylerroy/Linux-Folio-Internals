\documentclass[../main.tex]{subfiles}
\begin{document}
\chapter{从struct page到folio迁移指南}
\label{appendix:migration-guide}

\section{迁移概述}

从传统的\texttt{struct page}迁移到folio是一个渐进的过程,需要仔细规划和系统性的实施。本指南提供了完整的迁移路线图和最佳实践。

\section{迁移准备阶段}

\subsection{代码评估}

在开始迁移之前,需要对现有代码进行全面评估:

\begin{lstlisting}[language=bash,caption={代码评估脚本}]
#!/bin/bash
# 评估现有page使用情况的脚本

KERNEL_SRC="/lib/modules/$(uname -r)/build"
CODEBASE="."

echo "=== Page Usage Analysis ==="

# 统计page相关函数调用
echo "1. Page function call statistics:"
grep -r "page_" $CODEBASE --include="*.c" --include="*.h" | \
    grep -o "page_[a-zA-Z0-9_]*" | sort | uniq -c | sort -nr | head -20

# 统计复合页相关代码
echo -e "\n2. Compound page usage:"
grep -r "compound" $CODEBASE --include="*.c" --include="*.h" | wc -l

# 统计页缓存相关代码
echo -e "\n3. Page cache operations:"
grep -r "page_cache" $CODEBASE --include="*.c" --include="*.h" | wc -l

# 统计锁定相关代码
echo -e "\n4. Page locking operations:"
grep -r "lock_page\|unlock_page" $CODEBASE --include="*.c" --include="*.h" | wc -l
\end{lstlisting}

\subsection{依赖分析}

\begin{lstlisting}[language=C,caption={依赖分析工具}]
struct migration_dependency {
    const char *file_path;
    const char *function_name;
    enum dependency_type {
        DIRECT_PAGE_ACCESS,     /* 直接访问page结构 */
        PAGE_FUNCTION_CALL,     /* 调用page相关函数 */
        COMPOUND_PAGE_USE,      /* 使用复合页 */
        PAGE_CACHE_OPERATION,   /* 页缓存操作 */
        PAGE_LOCKING,          /* 页面锁定 */
    } type;
    bool has_folio_alternative; /* 是否有folio替代方案 */
    int complexity;            /* 迁移复杂度 (1-5) */
};

static struct migration_dependency *analyze_dependencies(const char *path)
{
    struct migration_dependency *deps = NULL;
    /* 实现依赖分析逻辑 */
    return deps;
}
\end{lstlisting}

\section{渐进式迁移策略}

\subsection{第一阶段:兼容层实现}

创建folio和page之间的兼容层:

\begin{lstlisting}[language=C,caption={兼容层实现}]
/* 兼容层头文件 */
#ifndef _LINUX_FOLIO_COMPAT_H
#define _LINUX_FOLIO_COMPAT_H

/* 条件编译支持 */
#ifdef CONFIG_FOLIO_MIGRATION
#define USE_FOLIO 1
#else
#define USE_FOLIO 0
#endif

/* 兼容宏定义 */
#if USE_FOLIO
#define page_operation(folio_ptr) folio_##folio_ptr
#define PAGE_TYPE struct folio *
#else
#define page_operation(page_ptr) page_##page_ptr
#define PAGE_TYPE struct page *
#endif

/* 混合操作函数 */
static inline void safe_page_get(void *page_or_folio)
{
#if USE_FOLIO
    if (is_folio_address(page_or_folio)) {
        folio_get((struct folio *)page_or_folio);
    } else {
        get_page((struct page *)page_or_folio);
    }
#else
    get_page((struct page *)page_or_folio);
#endif
}

#endif /* _LINUX_FOLIO_COMPAT_H */
\end{lstlisting}

\subsection{第二阶段:核心函数替换}

逐步替换核心操作函数:

\begin{lstlisting}[language=C,caption={核心函数替换示例}]
/* 旧的page操作函数 */
static int old_page_read_operation(struct page *page)
{
    lock_page(page);
    /* 执行读操作 */
    unlock_page(page);
    return 0;
}

/* 新的folio操作函数 */
static int new_folio_read_operation(struct folio *folio)
{
    folio_lock(folio);
    /* 执行读操作 */
    folio_unlock(folio);
    return 0;
}

/* 渐进式替换包装函数 */
static int progressive_read_operation(void *page_or_folio)
{
#ifdef CONFIG_FOLIO_MIGRATION
    if (is_folio_address(page_or_folio)) {
        return new_folio_read_operation((struct folio *)page_or_folio);
    }
#endif
    return old_page_read_operation((struct page *)page_or_folio);
}
\end{lstlisting}

\section{具体迁移步骤}

\subsection{步骤1:基础数据结构替换}

\begin{lstlisting}[language=C,caption={数据结构迁移}]
/* 旧的数据结构 */
struct my_page_cache {
    struct page *pages[MAX_PAGES];
    unsigned int count;
};

/* 新的数据结构 */
struct my_folio_cache {
    struct folio *folios[MAX_FOLIOS];
    unsigned int count;
};

/* 迁移函数 */
static int migrate_page_cache_to_folio(struct my_page_cache *old_cache,
                                     struct my_folio_cache *new_cache)
{
    int i;
    
    new_cache->count = old_cache->count;
    
    for (i = 0; i < old_cache->count; i++) {
        struct page *page = old_cache->pages[i];
        new_cache->folios[i] = page_folio(page);
        /* 增加folio引用计数 */
        folio_get(new_cache->folios[i]);
        /* 释放旧的page引用 */
        put_page(page);
    }
    
    return 0;
}
\end{lstlisting}

\subsection{步骤2:函数接口更新}

\begin{lstlisting}[language=C,caption={接口更新示例}]
/* 旧的函数接口 */
int process_pages(struct page **pages, int count);

/* 新的folio接口 */
int process_folios(struct folio **folios, int count);

/* 兼容性包装函数 */
int process_pages_compat(void **pages_or_folios, int count)
{
#ifdef CONFIG_FOLIO_MIGRATION
    if (is_folio_array(pages_or_folios, count)) {
        return process_folios((struct folio **)pages_or_folios, count);
    }
#endif
    return process_pages((struct page **)pages_or_folios, count);
}
\end{lstlisting}

\subsection{步骤3:内存管理优化}

\begin{lstlisting}[language=C,caption={内存管理迁移}]
/* 旧的内存分配 */
struct page *alloc_page_array(int count)
{
    struct page *pages = kmalloc(sizeof(struct page *) * count, GFP_KERNEL);
    int i;
    
    for (i = 0; i < count; i++) {
        pages[i] = alloc_page(GFP_KERNEL);
        if (!pages[i]) {
            /* 清理已分配的页面 */
            while (--i >= 0)
                __free_page(pages[i]);
            kfree(pages);
            return NULL;
        }
    }
    return pages;
}

/* 新的folio分配 */
struct folio *alloc_folio_array(int count)
{
    struct folio *folios = kmalloc(sizeof(struct folio *) * count, GFP_KERNEL);
    int i;
    
    for (i = 0; i < count; i++) {
        folios[i] = alloc_folio(GFP_KERNEL, 0);
        if (!folios[i]) {
            /* 清理已分配的folio */
            while (--i >= 0)
                folio_put(folios[i]);
            kfree(folios);
            return NULL;
        }
    }
    return folios;
}
\end{lstlisting}

\section{测试和验证}

\subsection{单元测试框架}

\begin{lstlisting}[language=C,caption={迁移测试框架}]
struct migration_test_case {
    const char *name;
    int (*test_function)(void);
    bool expect_success;
    const char *description;
};

static int test_folio_allocation(void)
{
    struct folio *folio;
    
    /* 测试基本分配 */
    folio = alloc_folio(GFP_KERNEL, 0);
    if (!folio)
        return -ENOMEM;
    
    /* 验证folio属性 */
    if (folio_order(folio) != 0) {
        folio_put(folio);
        return -EINVAL;
    }
    
    if (folio_nr_pages(folio) != 1) {
        folio_put(folio);
        return -EINVAL;
    }
    
    /* 清理 */
    folio_put(folio);
    return 0;
}

static int test_folio_conversion(void)
{
    struct page *page;
    struct folio *folio;
    
    /* 分配page并转换为folio */
    page = alloc_page(GFP_KERNEL);
    if (!page)
        return -ENOMEM;
    
    folio = page_folio(page);
    if (!folio) {
        __free_page(page);
        return -EINVAL;
    }
    
    /* 验证转换正确性 */
    if (folio_page(folio, 0) != page) {
        __free_page(page);
        return -EINVAL;
    }
    
    /* 清理 */
    __free_page(page);
    return 0;
}

static struct migration_test_case migration_tests[] = {
    {
        .name = "folio_allocation",
        .test_function = test_folio_allocation,
        .expect_success = true,
        .description = "测试folio基本分配功能"
    },
    {
        .name = "folio_conversion",
        .test_function = test_folio_conversion,
        .expect_success = true,
        .description = "测试page到folio的转换"
    },
    /* 更多测试用例... */
};
\end{lstlisting}

\subsection{性能基准测试}

\begin{lstlisting}[language=C,caption={性能基准测试}]
struct performance_benchmark {
    const char *name;
    unsigned long iterations;
    int (*old_function)(void);
    int (*new_function)(void);
    unsigned long old_time;
    unsigned long new_time;
    double improvement_ratio;
};

static unsigned long time_function(int (*func)(void), unsigned long iterations)
{
    struct timespec64 start, end;
    unsigned long i;
    int ret;
    
    ktime_get_ts64(&start);
    for (i = 0; i < iterations; i++) {
        ret = func();
        if (ret)
            break;
    }
    ktime_get_ts64(&end);
    
    return timespec64_to_ns(&end) - timespec64_to_ns(&start);
}

static void run_performance_benchmark(struct performance_benchmark *bench)
{
    pr_info("Running benchmark: %s\n", bench->name);
    
    /* 测试旧实现 */
    bench->old_time = time_function(bench->old_function, bench->iterations);
    
    /* 测试新实现 */
    bench->new_time = time_function(bench->new_function, bench->iterations);
    
    /* 计算性能提升 */
    if (bench->new_time > 0) {
        bench->improvement_ratio = (double)bench->old_time / bench->new_time;
    }
    
    pr_info("  Old implementation: %lu ns\n", bench->old_time);
    pr_info("  New implementation: %lu ns\n", bench->new_time);
    pr_info("  Performance improvement: %.2fx\n", bench->improvement_ratio);
}
\end{lstlisting}

\section{常见问题和解决方案}

\subsection{内存泄漏问题}

\begin{lstlisting}[language=C,caption={内存泄漏检测}]
/* 内存泄漏检测工具 */
static void check_folio_leaks(void)
{
    struct folio *folio;
    unsigned long leaked_count = 0;
    
    /* 遍历所有已分配的folio */
    spin_lock(&folio_list_lock);
    list_for_each_entry(folio, &global_folio_list, lru) {
        /* 检查引用计数异常 */
        if (atomic_read(&folio->_refcount) <= 0) {
            pr_err("Folio %p has invalid refcount: %d\n",
                   folio, atomic_read(&folio->_refcount));
            leaked_count++;
        }
        
        /* 检查锁定状态异常 */
        if (folio_test_locked(folio) && 
            !folio_test_owner(folio, current)) {
            pr_warn("Folio %p is locked by another process\n", folio);
        }
    }
    spin_unlock(&folio_list_lock);
    
    if (leaked_count > 0) {
        pr_err("Found %lu potentially leaked folios\n", leaked_count);
    }
}
\end{lstlisting}

\subsection{竞态条件处理}

\begin{lstlisting}[language=C,caption={竞态条件防护}]
/* 安全的folio操作模板 */
static int safe_folio_operation(struct folio *folio, 
                               int (*operation)(struct folio *))
{
    int ret;
    
    /* 预检查 */
    if (!folio)
        return -EINVAL;
    
    /* 原子锁定 */
    if (folio_lock_killable(folio))
        return -EINTR;
    
    /* 再次验证folio状态 */
    if (folio_test_hwpoison(folio)) {
        ret = -EHWPOISON;
        goto unlock;
    }
    
    /* 执行操作 */
    ret = operation(folio);
    
unlock:
    folio_unlock(folio);
    return ret;
}

/* 使用示例 */
static int my_folio_operation(struct folio *folio)
{
    return safe_folio_operation(folio, do_actual_operation);
}
\end{lstlisting}

\section{回滚策略}

\subsection{配置开关实现}

\begin{lstlisting}[language=C,caption={运行时配置开关}]
/* 配置管理结构 */
struct folio_migration_config {
    bool enabled;                    /* folio功能是否启用 */
    bool fallback_to_page;          /* 失败时是否回退到page */
    unsigned int compatibility_mode; /* 兼容模式级别 */
    bool debug_mode;                /* 调试模式 */
};

static struct folio_migration_config migration_config = {
    .enabled = IS_ENABLED(CONFIG_FOLIO_DEFAULT_ENABLED),
    .fallback_to_page = true,
    .compatibility_mode = 2,  /* 中等兼容性 */
    .debug_mode = false,
};

/* 运行时配置接口 */
static int set_folio_migration_enabled(const char *val, 
                                     const struct kernel_param *kp)
{
    int ret;
    bool old_state = migration_config.enabled;
    
    ret = param_set_bool(val, kp);
    if (ret)
        return ret;
    
    /* 状态变更处理 */
    if (old_state != migration_config.enabled) {
        if (migration_config.enabled) {
            pr_info("Folio migration enabled\n");
            /* 执行启用初始化 */
            folio_migration_init();
        } else {
            pr_info("Folio migration disabled\n");
            /* 执行禁用清理 */
            folio_migration_cleanup();
        }
    }
    
    return 0;
}

static const struct kernel_param_ops folio_enabled_ops = {
    .set = set_folio_migration_enabled,
    .get = param_get_bool,
};

module_param_cb(folio_enabled, &folio_enabled_ops, 
                &migration_config.enabled, 0644);
\end{lstlisting}

\subsection{渐进式回滚机制}

\begin{lstlisting}[language=C,caption={渐进式回滚}]
/* 回滚管理器 */
struct rollback_manager {
    struct list_head rollback_points;
    unsigned int current_level;
    bool rollback_in_progress;
};

struct rollback_point {
    struct list_head list;
    unsigned int level;
    void (*rollback_function)(void);
    void *rollback_data;
};

static int register_rollback_point(unsigned int level,
                                 void (*rollback_func)(void),
                                 void *data)
{
    struct rollback_point *point;
    
    if (rollback_mgr.rollback_in_progress)
        return -EBUSY;
    
    point = kmalloc(sizeof(*point), GFP_KERNEL);
    if (!point)
        return -ENOMEM;
    
    point->level = level;
    point->rollback_function = rollback_func;
    point->rollback_data = data;
    
    list_add_tail(&point->list, &rollback_mgr.rollback_points);
    rollback_mgr.current_level = level;
    
    return 0;
}

static void execute_rollback(void)
{
    struct rollback_point *point, *tmp;
    
    rollback_mgr.rollback_in_progress = true;
    
    /* 按逆序执行回滚 */
    list_for_each_entry_safe_reverse(point, tmp, 
                                   &rollback_mgr.rollback_points, list) {
        if (point->rollback_function) {
            point->rollback_function();
        }
        list_del(&point->list);
        kfree(point);
    }
    
    rollback_mgr.rollback_in_progress = false;
    rollback_mgr.current_level = 0;
}
\end{lstlisting}

\section{总结}

folio迁移是一个复杂但值得的投资。通过遵循本指南的步骤和最佳实践,可以确保迁移过程的平稳和成功。关键要点包括:

\begin{itemize}
    \item 采用渐进式迁移策略,避免一次性大规模改动
    \item 实现完善的兼容层和回滚机制
    \item 建立全面的测试和监控体系
    \item 保持代码的可维护性和可读性
    \item 充分利用folio提供的性能和功能优势
\end{itemize}

成功的迁移不仅能够提升系统性能,还能为未来的内核发展奠定坚实基础。


\end{document}
